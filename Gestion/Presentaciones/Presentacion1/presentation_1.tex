%%%%%%%%%%%%%%%%%%%%%%%%%%%%%%%%%%%%%%%%%
% Beamer Presentation
% LaTeX Template
% Version 1.0 (10/11/12)
%
% This template has been downloaded from:
% http://www.LaTeXTemplates.com
%
% License:
% CC BY-NC-SA 3.0 (http://creativecommons.org/licenses/by-nc-sa/3.0/)
%
%%%%%%%%%%%%%%%%%%%%%%%%%%%%%%%%%%%%%%%%%

%----------------------------------------------------------------------------------------
%	PACKAGES AND THEMES
%----------------------------------------------------------------------------------------

\documentclass{beamer}

\usepackage[spanish,USenglish]{babel} % espanol, ingles
\usepackage[utf8]{inputenc} % acentos sin codigo

\mode<presentation> {

% The Beamer class comes with a number of default slide themes
% which change the colors and layouts of slides. Below this is a list
% of all the themes, uncomment each in turn to see what they look like.

%\usetheme{default}
%\usetheme{AnnArbor}
%\usetheme{Antibes}
%\usetheme{Bergen}
%\usetheme{Berkeley}
%\usetheme{Berlin}
%\usetheme{Boadilla}
%\usetheme{CambridgeUS}
%\usetheme{Copenhagen}
%\usetheme{Darmstadt}
%\usetheme{Dresden}
%\usetheme{Frankfurt}
%\usetheme{Goettingen}
%\usetheme{Hannover}
%\usetheme{Ilmenau}
%\usetheme{JuanLesPins}
%\usetheme{Luebeck}
\usetheme{Madrid}
%\usetheme{Malmoe}
%\usetheme{Marburg}
%\usetheme{Montpellier}
%\usetheme{PaloAlto}
%\usetheme{Pittsburgh}
%\usetheme{Rochester}
%\usetheme{Singapore}
%\usetheme{Szeged}
%\usetheme{Warsaw}

% As well as themes, the Beamer class has a number of color themes
% for any slide theme. Uncomment each of these in turn to see how it
% changes the colors of your current slide theme.

%\usecolortheme{albatross}
%\usecolortheme{beaver}
%\usecolortheme{beetle}
%\usecolortheme{crane}
%\usecolortheme{dolphin}
%\usecolortheme{dove}
%\usecolortheme{fly}
%\usecolortheme{lily}
%\usecolortheme{orchid}
%\usecolortheme{rose}
%\usecolortheme{seagull}
%\usecolortheme{seahorse}
%\usecolortheme{whale}
%\usecolortheme{wolverine}

%\setbeamertemplate{footline} % To remove the footer line in all slides uncomment this line
%\setbeamertemplate{footline}[page number] % To replace the footer line in all slides with a simple slide count uncomment this line

%\setbeamertemplate{navigation symbols}{} % To remove the navigation symbols from the bottom of all slides uncomment this line
}

\usepackage{graphicx} % Allows including images
\usepackage{booktabs} % Allows the use of \toprule, \midrule and \bottomrule in tables

%----------------------------------------------------------------------------------------
%	TITLE PAGE
%----------------------------------------------------------------------------------------

\title[]{Propuesta de proyecto} % The short title appears at the bottom of every slide, the full title is only on the title page

\author{Sworkware Consultory - Designing Sales} % Your name
\institute[IPN] % Your institution as it will appear on the bottom of every slide, may be shorthand to save space
{
Escuela Superior de Cómputo \\ % Your institution for the title page
\medskip
\textit{gabriela.moreno.glez@gmail.com} % Your email address
}
\date{\today} % Date, can be changed to a custom date

\begin{document}

\begin{frame}
\titlepage % Print the title page as the first slide
\end{frame}

\begin{frame}
\frametitle{índice} % Table of contents slide, comment this block out to remove it
\tableofcontents % Throughout your presentation, if you choose to use \section{} and \subsection{} commands, these will automatically be printed on this slide as an overview of your presentation
\end{frame}

%----------------------------------------------------------------------------------------
%	PRESENTATION SLIDES
%----------------------------------------------------------------------------------------

%------------------------------------------------
\section{Contexto del problema} % Sections can be created in order to organize your presentation into discrete blocks, all sections and subsections are automatically printed in the table of contents as an overview of the talk
%------------------------------------------------

\begin{frame}
\frametitle{Contexto del problema}

En el Instituto Politécnico Nacional se imparten diversas unidades de aprendizaje de acuerdo a la escuela y a la carrera que se esté cursando. Algunas de ellas pertenecen al área de Electrónica como Análisis de Circuitos e Instrumentación, por lo que los estudiantes requieren adquirir los componentes electrónicos para realizar sus prácticas, sin embargo suele ser costoso o difícil de haya algunos de éstos.

\end{frame}

%------------------------------------------------

\begin{frame}
\frametitle{Contexto del problema}
Los problemas más comunes al ir a comprar un componente electrónico son:
\begin{itemize}
\item Ninguna tienda tiene el componente
\item El componente ha dejado de ser fabricado
\item El componente es muy caro
\item El componente no tiene especificaciones
\item El componente viene dañado y no posee garantía
\end{itemize}
\end{frame}

%------------------------------------------------

\section{Problemática}

\begin{frame}
\frametitle{Problemática}
\begin{block}{Componentes que no se vuelven a usar}
Muchas veces los estudiantes realizan la inversión para comprar sus materiales y nunca los vuelven a usar ni en el resto de su carrera ni cuando comienzan a laborar.
\end{block}

\begin{block}{Ineficiencia al localizar un componente}
Los componentes comprados hace años por otros estudiantes o jamás se ponen en venta o simplemente los publican en sitios poco convenientes como Facebook y la publicación jamás es vista por el público que está interesado.
\end{block}

\begin{block}{Inexistencia de la donación de componentes}
La única forma en la que algunos estudiantes regalan sus componentes es cuando conocen a la persona y le solicita que se los preste, a lo que el estudiante prefiere regalarlos porque no los va a volver a usar.
\end{block}
\end{frame}

\begin{frame}
\frametitle{Problemática}
Así, nuestra pregunta es: \\

¿De qué manera podemos apoyar a los estudiantes de las escuelas del IPN en Zacatenco a comprar o donar componentes electrónicos para su uso en las Unidades de Aprendizaje de electrónica?
\end{frame}

%------------------------------------------------

%\begin{frame}
%\frametitle{Multiple Columns}
%\begin{columns}[c] % The "c" option specifies centered vertical alignment while the "t" option is used for top vertical alignment

%\column{.45\textwidth} % Left column and width
%\textbf{Heading}
%\begin{enumerate}
%\item Statement
%\item Explanation
%\item Example
%\end{enumerate}

%\column{.5\textwidth} % Right column and width
%Lorem ipsum dolor sit amet, consectetur adipiscing elit. Integer lectus nisl, ultricies in feugiat rutrum, porttitor sit amet augue. Aliquam ut tortor mauris. Sed volutpat ante purus, quis accumsan dolor.

%\end{columns}
%\end{frame}

%------------------------------------------------
\section{Propuesta de solución}
%------------------------------------------------

\begin{frame}
\frametitle{Propuesta de solución}
Realizar un sistema que permita la administración de ventas y donaciones de componentes electrónicos entre los estudiantes de escuelas superiores del IPN en Zacatenco.\\

Las características del sistema son:
\begin{itemize}
\item  Poseerá un repositorio de material electrónico.
\item Los estudiantes podrán crear un perfil en el sistema.
\item  Se tendrá el link al perfil de Facebook de los estudiantes con un perfil en el sistema
\item Los estudiantes podrán donar, vender o comprar componentes
\item  Los componentes tendrán la siguiente información: nombre, cantidad, link a la datasheet (opcional) y si se va a donar o a vender
\item  Se podrán calificar los perfiles de los usuarios
\item  El sistema será un sitio web.
\item Se empleará BootStrap para su compatibilidad con dispositivos móviles
\end{itemize}
%\begin{table}
%\begin{tabular}{l l l}
%\toprule
%\textbf{Treatments} & \textbf{Response 1} & \textbf{Response 2}\\
%\midrule
%Treatment 1 & 0.0003262 & 0.562 \\
%Treatment 2 & 0.0015681 & 0.910 \\
%Treatment 3 & 0.0009271 & 0.296 \\
%\bottomrule
%\end{tabular}
%\caption{Table caption}
%\end{table}
\end{frame}

%------------------------------------------------

%\begin{frame}
%\frametitle{Propuesta de solución}
%\begin{theorem}[Mass--energy equivalence]
%$E = mc^2$
%\end{theorem}
%\end{frame}

%------------------------------------------------

%\begin{frame}[fragile] % Need to use the fragile option when verbatim is used in the slide
%\frametitle{Verbatim}
%\begin{example}[Theorem Slide Code]
%\begin{verbatim}
%\begin{frame}
%\frametitle{Theorem}
%\begin{theorem}[Mass--energy equivalence]
%$E = mc^2$
%\end{theorem}
%\end{frame}\end{verbatim}
%\end{example}
%\end{frame}

%------------------------------------------------

%\begin{frame}
%\frametitle{Figure}
%Uncomment the code on this slide to include your own image from the same directory as the template .TeX file.
%\begin{figure}
%\includegraphics[width=0.8\linewidth]{test}
%\end{figure}
%\end{frame}

%------------------------------------------------

%\begin{frame}[fragile] % Need to use the fragile option when verbatim is used in the slide
%\frametitle{Citation}
%An example of the \verb|\cite| command to cite within the presentation:\\~

%This statement requires citation \cite{p1}.
%\end{frame}

%------------------------------------------------

%\begin{frame}
%\frametitle{References}
%\footnotesize{
%\begin{thebibliography}{99} % Beamer does not support BibTeX so references must be inserted manually as below
%\bibitem[Smith, 2012]{p1} John Smith (2012)
%\newblock Title of the publication
%\newblock \emph{Journal Name} 12(3), 45 -- 678.
%}
%\end{frame}

%------------------------------------------------

\begin{frame}
\Huge{\centerline{Gracias por su atención}}
\end{frame}

%----------------------------------------------------------------------------------------

\end{document} 