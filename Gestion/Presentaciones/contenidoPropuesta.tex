\section{Contexto del problema}

Muchas de las unidades de aprendizaje que se imparten a nivel superior dentro del Instituto Polit\'ecnico Nacional están enfocadas a el \'area de electr\'onica. Cada una de esas materias son impartidas en distintas unidades académicas, tales como, ESCOM, ESIME, UPIITA y UPIICSA, todas ellas con enfoque f\'isico matem\'aticas.

Todas estas unidades de aprendizaje est\'an relacionadas de una u otra manera, por lo que la elaboración de las prácticas son muy similares en cuanto a componentes electr\'onicos se refiere. Lo anterior, nos obliga a pensar que la adquisición de los componentes llega a ser muy similar entre los alumnos que las cursan, as\'i como tambi\'en, el olvido o abandono de los mismos una vez finalizado el semestre.

Ese material electrónico que se adquirió para la elaboración de prácticas, muchas de las veces se ocupa una sola vez o incluso, en ocasiones, ni siquiera se utiliza.

Así, en el mejor de los casos, ese material se regala a otros compañeros que lo necesitan o se vende, pero si no es así, \'este se queda arrumbado en casa. 

Entonces, ¿cómo hacer para que esos componentes sean de utilidad todavía?, ¿cómo saber qué componentes tienen mis demás compañeros?, ¿qué puedo hacer con mis componentes electrónicos del semestre pasado?.

Aunado a eso, muchas de las veces el material electr\'onico es complicado de conseguir o incluso bastante caro, por lo que, los alumnos optan por buscar material de segunda mano. Esto a través de compañeros que cursaron la materia en niveles anteriores o que quizá tienen ese componente de sobra. 

As\'i, los medios por los cuales se pueden adquirir o publicar dichos componentes son en su mayor parte por medios electr\'onicos, Facebook es un gran ejemplo de ello y hoy en d\'ia el medio m\'as habitual para esto.

\section{Problem\'atica}
\begin{itemize}
    \item Componentes que no se vuelven hacer.
    \item Ineficiencia al localizar un componente.
    \item Inexistencia de la donaci\'on de componentes.
\end{itemize}

Por lo tanto,

¿De qu\'e manera podemos apoyar a los estudiantes de las escuelas del IPN en Zacatenco a comprar o donar componentes electr\'onicos para su uso en las Unidades de Aprendizaje afines a la electr\'onica?

\section{Propuesta de soluci\'on}
Realizar una aplicaci\'on web que permita la publicaci\'on de ventas y donaciones de componentes electr\'onicos entre los estudiantes de escuelas superiores del IPN en Zacatenco.\\
Las caracter\'isticas del sistema son:\\ \\
\begin{itemize}
    \item Poseer\'a un repositorio de material electr\'onico.
    \item Los estudiantes podr\'an crear un perfil en el sistema.
    \item Se tendr\'a el link al perfil de Facebook de los estudiantes con un perfil creado en la aplicaci\'on.
    \item Los estudiantes podr\'an donar, vender o comprar componentes.
    \item Los componentes tendr\'an la siguiente informaci\'on: nombre, cantidad, link a la datasheet (opcional) y la opci\'on para donar o vender.
    \item Se podr\'an calificar los perfiles de los usuarios.
    \item Buscar y remplazar código.
    \item Se emplear\'a BootStrap como framework de diseño para la responsividad de la aplicaci\'on.
\end{itemize}


\section{Referencias}
\begin{enumerate}
    \item \url{http://www.ipn.mx/educacionsuperior/Paginas/modalidad-escolarizada.aspx}
    \item \url{https://es-la.facebook.com/}
\end{enumerate}











 