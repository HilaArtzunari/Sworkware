%%%%%%%%%%%%%%%%%%%%%%%%%%%%%%%%%%%%%%%%%%%%%%%%%%%%%%%%%%%%%%
%  Documento raíz: manual.tex
%  Contenido: Capítulo ``El paquete cdtTheme''
%  Responsable: Ulises Vélez Saldaña
%%%%%%%%%%%%%%%%%%%%%%%%%%%%%%%%%%%%%%%%%%%%%%%%%%%%%%%%%%%%%%


%---------------------------------------------------------
\section{Paquetes que incluye}

\begin{Cdescription}
	\item[Paquete:] {\tt inputenc}
	\item[Propósito:] Poder utilizar caracteres especiales dentro de los documentos de \LaTeX así como correctores ortográficos.
	\item[Configuración:] El paquete está configurado para utilizar la codificación UTF-8.
	\item[Consideraciones adicionales:] Revisar el manual del paquete en \url{http://tug.org}.
\end{Cdescription}

\begin{Cdescription}
	\item[Paquete:] {\tt babel}
	\item[Propósito:] Configurar el idioma del conjunto de constantes que utiliza \LaTeX en los documentos compilados 
	\item[Configuración:] Por defecto está configurado a español.
	\item[Consideraciones adicionales:] Revisar el manual del paquete en \url{http://tug.org}.
\end{Cdescription}

\begin{Cdescription}
	\item[Paquete:] {\tt wasysym}
	\item[Propósito:] Ofrecer un conjunto de caracteres.
	\item[Configuración:] No aplica.
	\item[Consideraciones adicionales:] Revisar la tabla de caracteres que ofrece este paquete en \url{http://tug.org}.
\end{Cdescription}

\begin{Cdescription}
	\item[Paquete:] {\tt marvosym}
	\item[Propósito:] Ofrecer un conjunto de caracteres.
	\item[Configuración:] No aplica.
	\item[Consideraciones adicionales:] Revisar la tabla de caracteres que ofrece este paquete en \url{http://tug.org}.
\end{Cdescription}

\begin{Cdescription}
	\item[Paquete:] {\tt cmbright}
	\item[Propósito:] Ofrece la fuente utilizada por defecto en toda la documentación de la CDT.
	\item[Configuración:] No aplica.
	\item[Consideraciones adicionales:] Revisar el manual del paquete en \url{http://tug.org}.
\end{Cdescription}

\begin{Cdescription}
	\item[Paquete:] {\tt graphicx}
	\item[Propósito:] Paquete utilizado para incluir las imágenes en los documentos.
	\item[Configuración:] Ninguna.
	\item[Consideraciones adicionales:] Revisar el manual del paquete en \url{http://tug.org}.
\end{Cdescription}

\begin{Cdescription}
	\item[Paquete:] {\tt xcolor}
	\item[Propósito:] Colores para texto y tablas.
	\item[Configuración:] {\tt usenames} para usar los mnemonicos de los colores y {\tt dvipsnames} para utilizar los nombres de DVI y PostScript.
	\item[Consideraciones adicionales:] Revisar el manual del paquete en \url{http://tug.org}.
\end{Cdescription}

\begin{Cdescription}
	\item[Paquete:] {\tt wallpaper}
	\item[Propósito:] Colocar imágenes de fondo en los documentos.
	\item[Configuración:] No aplica
	\item[Consideraciones adicionales:] Revisar el manual del paquete en \url{http://tug.org}.
\end{Cdescription}

\begin{Cdescription}
	\item[Paquete:] {\tt fancyhdr}%Descripción redactada por EVELYN.
	\item[Propósito:] Paquete utilizado para personalizar los encabezados y pies de página.
	\item[Configuración:]
	\item[Consideraciones adicionales:]Revisar el manual del paquete en \url{http://tug.org}.
\end{Cdescription}

\begin{Cdescription}%Descripción redactada por EVELYN.
	\item[Paquete:] {\tt geometry}
	\item[Propósito:] Paquete utilizado para definir el diseño de página.% con los parámetros intuitivos
	\item[Configuración:]
	\item[Consideraciones adicionales:]
\end{Cdescription}

\begin{Cdescription}
	\item[Paquete:] {\tt fncychap}%Descripción redactada por EVELYN.
	\item[Propósito:] Paquete utilizado para definir el estilo  de los encabezados de cada capítulo
	\item[Configuración:]
	\item[Consideraciones adicionales:] %http://sistemas.fciencias.unam.mx/~misraim/fancy.pdf
\end{Cdescription}

\begin{Cdescription}%Descripción redactada por EVELYN.
	\item[Paquete:] {\tt url}
	\item[Propósito:] Paquete utilizado para realizar saltos de línea en ciertos caracteres o combinaciones de caracteres.
	\item[Configuración:]
	\item[Consideraciones adicionales:] %http://www.ctan.org/tex-archive/macros/latex/contrib/url.
\end{Cdescription}

\begin{Cdescription}%Descripción redactada por EVELYN.
	\item[Paquete:] {\tt hyperref} Hyperref is a TeX package for making documents with live links in PDF and HTML output formats. 
	\item[Propósito:] Paquete utlizado permite la creación de documentos interactivos.
	\item[Configuración:]
	\item[Consideraciones adicionales:]
\end{Cdescription}

%---------------------------------------------------------
\section{Variables definidas}

\begin{Cdescription}
	\item[Variable:] {\tt colorPrincipal}
	\item[Tipo:] Nombre de color.
	\item[Propósito:]
	\item[Uso:]
	\item[Ejemplo:]
\end{Cdescription}

\begin{Cdescription}
	\item[Variable:] {\tt colorSecundario}
	\item[Tipo:] Nombre de color.
	\item[Propósito:]
	\item[Uso:]
	\item[Ejemplo:]
\end{Cdescription}

\begin{Cdescription}
	\item[Variable:] {\tt colorGris}
	\item[Tipo:] Nombre de color.
	\item[Propósito:]
	\item[Uso:]
	\item[Ejemplo:]
\end{Cdescription}

\begin{Cdescription}
	\item[Variable:] {\tt colorAgua}
	\item[Tipo:] Nombre de color.
	\item[Propósito:]
	\item[Uso:]
	\item[Ejemplo:]
\end{Cdescription}

\begin{Cdescription}
	\item[Variable:] {\tt $\backslash$cdtTelefono}
	\item[Tipo:] Comando.
	\item[Propósito:]
	\item[Uso:]
	\item[Ejemplo:]
\end{Cdescription}

\begin{Cdescription}
	\item[Variable:] {\tt $\backslash$cdtDireccion}
	\item[Tipo:] Comando.
	\item[Propósito:]
	\item[Uso:]
	\item[Ejemplo:]
\end{Cdescription}

\begin{Cdescription}
	\item[Variable:] {\tt $\backslash$cdtCorreo}
	\item[Tipo:] Comando.
	\item[Propósito:]
	\item[Uso:]
	\item[Ejemplo:]
\end{Cdescription}

%---------------------------------------------------------
\section{Comandos}

\begin{Cdescription}
	\item[Comando:] {\tt $\backslash$setDireccion}
	\item[Propósito:] Establecer o cambiar la dirección de la organización.
	\item[Ejemplo:] $\backslash$setDireccion\{Av. Juan de Dios Bátiz esq. Miguel Othón de Mendizabal S/N Col. Lindavista, GAM, D. F.\}	
\end{Cdescription}


\begin{Cdescription}
	\item[Comando:] {\tt $\backslash$setTelefono}
	\item[Propósito:] Definir la dirección de correo de la organización.
	\item[Ejemplo:] $\backslash$setTelefono\{5544332211\}	
\end{Cdescription}

\begin{Cdescription}
	\item[Comando:] {\tt $\backslash$setCorreo}
	\item[Propósito:] Definir la dirección de correo de la organización.
	\item[Ejemplo:] $\backslash$setCorreo\{ulises.velez@gmail.com\}	
\end{Cdescription}


%---------------------------------------------------------
\section{Recursos relacionados}

\begin{Cdescription}
	\item[Recurso:] {\tt cdt/theme/headerPar}
	\item[Tipo:] Imagen en formato PNG.
	\item[Propósito:]
	\item[Uso:]
\end{Cdescription}

\begin{Cdescription}
	\item[Recurso:] {\tt cdt/theme/headerInp}
	\item[Tipo:] Imagen en formato PNG.
	\item[Propósito:]
	\item[Uso:]
\end{Cdescription}

\begin{Cdescription}
	\item[Recurso:] {\tt cdt/theme/logoPar}
	\item[Tipo:] Imagen en formato PNG.
	\item[Propósito:]
	\item[Uso:]
\end{Cdescription}

\begin{Cdescription}
	\item[Recurso:] {\tt cdt/theme/logoInp}
	\item[Tipo:] Imagen en formato PNG.
	\item[Propósito:]
	\item[Uso:]
\end{Cdescription}

%---------------------------------------------------------
\section{Configuraciones específicas}

\TODO[Describir los márgenes, dimensiones y los cambios de color a viñetas y secciones.]
