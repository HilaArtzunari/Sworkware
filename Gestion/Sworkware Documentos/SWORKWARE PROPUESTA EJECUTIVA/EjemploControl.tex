\documentclass[10pt]{book}
\usepackage{cdt/cdtControl}

%%%%%%%%%%%%%%%%%%%%%%%%%%%%%%%%%%%%%%%%%%%%%%%%%%%%%%%%%%%%%%%%
% Datos del proyecto

\cdtOrganizacion[CNSNS--SENER]{Comisión Nacional de Seguridad Nuclear y Salvaguardias, SENER}

\cdtAutor{Coordinación de Desarrollo Tecnológico, IPN}

\cdtSistema[REPO]{Subsistema de Repositorio de Información}

\cdtProyecto[234412, IPN-23.13-SCOR2]{Sistema de Control Radiológico Versión 2.0.}

\cdtDocumento{Propuesta}{Propuesta técnica}{\DRAFT{\today}} %\RELEASE{1.0}

\cdtEntregable{E1}{Entregable 1}

% Descomentar y establecer la fecha cuando se desee congelar la fecha del documento.
%\cdtFecha{12 de Abril de 2013}

%%%%%%%%%%%%%%%%%%%%%%%%%%%%%%%%%%%%%%%%%%%%%%%%%%%%%%%%%%%%%%%%

\begin{document}

%=========================================================
% Portada
\thispagestyle{empty}

\maketitle

%=========================================================
% Indices del documento
\frontmatter
\tableofcontents
\listoffigures
\listoftables
\mainmatter

% Para esconder la información del documentador se descomenta el \hideControlVersion
%\cdtHideControlVersion
%\cdtHideInstrucciones

%=========================================================
\chapter{Introducción}

\cdtInstrucciones{Se pueden colocar instrucciones para indicar lo que se debe escribir dentro de los formatos.}

\begin{techCard}[version=1.0, author=Edgar, status=\cdtStAccepted, revisor=Juan]
	\cItem{Actualizado el}{12/10/2016}
\end{techCard}

\refElem{UC1}

\begin{revCard}{0.1}{Juan Carlos}{12 de abril, 2013}
	\rItem{PC1}{Verificar y corregir ortografía.}{\TODO[1 días].}
	\rItem{PC2}{Actualizar el diagrama de la BD.}{\TOCHK[15/04/2013].}
	\rItem{PC3}{Agregar las definiciones para los términos del negocio.}{\DONE[20/04/2013].}
\end{revCard}


%==========================================================
\clossing
\end{document}
