%=========================================================
\chapter{Análisis del problema}
\label{cap:analisis}

	Este capítulo contiene la justificación del proyecto. Presenta una breve descripción del club, un análisis del problema identificando sus principales causas y termina con la estimación de las consecuencias más importantes a mediano y largo plazo. Para este análisis se utilizaron las técnicas de ``lluvia de ideas'', ``Diagrama de ishikahua'' y ``Análisis FODA''.

%----------------------------------------------------------
\section{Términos del negocio}
\label{sec:terminosDeNegocio}
% Status:
% \cdtStRevision cuando el autor ya terminó de trabajar
% \cdtStRevised  cuando el revisor acepta la redacción
% \cdtStEnded    cuando la revisión fue aprobada por el líder de proyecto
% \cdtStAccepted cuando el Cliente aprobó la redacción
\begin{brGlosario}[
	version=1.0, 
	author=Gabriela Moreno, 
	revisor=Ulises Pérez, 
	status =\cdtStEdition
]
	\brTermType{tCoordinador}{Coordinador:} ...
	\brTermType{tIntegrante}{Integrante:} ...
	\brTermType{tParticipante}{Participante:} ...
	\brTermType{tActividad}{Actividad:} ...

%	\brTermLiteral{tAutomovil}{Automóvil:}{tVehiculo}{Vehículo} De cuatro ruedas con capacidad de 5 a 9 personas. 
%	\brTermType{tCliente}{Cliente:} Se refiere a todas las personas físicas y morales que \cdtRef{tRenta}{rentan} o han rentado un \cdtRef{tVehiculo}{vehículo}.
%	
%	\brTermLiteral{tDirector}{Director:}{tEmpleado}{Empleado} Es el empleado que tiene mayor rango de todos y no tiene superior, a diferencia de los demás.
%	
%	\brTermType{tEmpleado}{Empleado:} Se refiere a cualquier persona que labore en la empresa.
%	
%	\brTermClock{tChecador}{Checador:}{Hora de entrada y salida de un \cdtRef{tEmpleado}{empleado}.}{Una vez al día para la entrada y otra para la salida durante los días laborales.}
%	
%	\brTermLiteral{tMotocicleta}{Motocicleta:}{tVehiculo}{Vehículo} De dos ruedas con capacidad para una personas. 
%
%	\brTermType{tRenta}{Renta:} Se refiere al servicio que ofrece la empresa para prestar \cdtRef{tVehiculo}{vehículos} a los \cdtRef{tCliente}{clientes} por un tiempo definido.
%	
%	\brTermType{tVehiculo}{Vehiculo:} Se refiere a los automóviles y motocicletas que la empresa usa para dar el servicio de renta a los \cdtRef{tCliente}{clientes}.
%	
%	\brTermSensor{tVelocimetro}{Velocímetro:}{Velocidad de un Vehículo.}{Kilometros/hora.}{Constantemente siempre que el \cdtRef{tVehiculo}{vehículo} esté encendido.}
\end{brGlosario}



%---------------------------------------------------------
\section{Descripción del contexto}

	El Club de Bio-Robótica\footnote{En adelante solo ``Club''.} es un grupo estudiantil de la Escuela Superior de Cómputo, el cual se dedica a realizar actividades académicas.
	
	Este Club está conformado por alumnos y profesores de la ESCOM, por profesores de otras escuelas y por público en general. Cada uno de estos \cdtRef{tParticipante}{Participantes} colaboran y participan en las actividades de distintas formas. Por consiguiente también deberían poder acceder a la información del grupo a diferentes niveles.
	
	El Club realiza diversas actividades: participación en concursos, participación en congresos, talleres y actividades lúdicas. En todas esas actividades el Club realiza convocatorias, organiza los eventos, consigue recursos, genera materiales y publica resultados con fotos y videos.	
	
	La principal motivación de los integrantes del Club es la liberación de créditos de materias electivas. Lo cual se realiza mediante su participación en las actividades de Club. El coordinador del Club reporta cada semestre a los integrantes activos y la participación de cada uno para la generación de sus constancias.
	
%---------------------------------------------------------
\section{Descomposición del problema}

	Actualmente el Club cuenta con una página desactualizada en la que publica toda su información, de la cual podemos decir que:
	
\begin{itemize}
	\item Hay mucha información que se quisiera proteger.
	\item Hay información por actualiza o agregar.
	\item No se tienen históricos.
	\item Su navegación es difícil.
	\item Se desea un nuevo diseño y mayor identidad del Club.
	\item No hay integración entre el sitio y las redes sociales del Club.
	\item Se desea publicar información de eventos y un esquema mas avanzado de privilegios de acceso a la plataforma.
	\item Problemas para llevar un registro actualizado de integrantes activos y su participación en las actividades del Club.
\end{itemize}

%---------------------------------------------------------
\section{Análisis de causas y consecuencias}

	Gran parte de la problemática se debe a que la página no tiene una buena estructura y en su mayoría es solo información en HTML. No tiene una Base de datos en la que se vaya agregando la información y no hay un responsable de comunicaciones en el grupo que se dedique a actualizar la información o a publicar eventos, fotos y videos.
	
	Como consecuencia, las actividades del Club pueden no tener el impacto deseado, una baja participación o aprovechamiento de los recursos disponibles. También es posible que se cometan errores al momento de reportar las participaciones de los integrantes para la generación de sus constancias o se tenga desactualizada la información de los participantes.

%---------------------------------------------------------
\section{Síntesis y propuesta de solución}

	Tras un estudio realizado sugerimos lo siguiente:

\begin{itemize}
	\item El uso de un canal de YouTube para el Club y vincularlo a la página. Esto debido a que los videos suelen crear grandes demandas de almacenamiento y es preferible utilizar una red social y plataforma que ya resuelve esta problemática
	\item Revisión e integración de redes sociales que ayuden a publicar mejor los eventos y actividades del grupo, así como su vinculación con la página.
	\item Reestructuración y rediseño de la página para mejorar su navegabilidad, organización y actualizar la información en la misma.
	\item Llevar un registro de las participaciones y lista de integrantes activos en cada semestre.
\end{itemize}

% - - - - - - - - - - - - - - - - - - - - - - - - - - - - - 
\subsection{Propuesta de solución}

	Con base en lo analizado, se propone la reestructuración del sitio mediante cuatro etapas iniciales:
	
\begin{description}
	\item[Análisis y recuperación:] Revisar lo que hay de información y evaluar la situación:	
	\begin{itemize}
 		\item Estudiar la información actual en el servidor.
		\item Recuperar dicha información y toda la que exista y pueda servir.
		\item Integrar toda la información y definir que sirve y que es obsoleto o irrelevante.
	\end{itemize}
	\item[Reestructuración:] Definir una nueva estructura, una nueva imagen y construir dicho sitio.	
    \begin{itemize}
    	\item Estudiar otros grupos de investigación y reestructurar la información del sitio.
		\item Definir tres diseños gráficos para el sitio para su presentación y autorización por parte del Club.
		\item Integrar Bootstrap para su compatibilidad con dispositivos móviles.
		\item Construir el sitio con el nuevo diseño.
    \end{itemize}
	\item[Migración de galerías y redes sociales:] Abrir un canal en YoutTube, FaceBook y otra red social que permita subir todas las fotos e imágenes seleccionadas e integrarlas con el sitio.	
    \begin{itemize}
    	\item Abrir canal de Youtube y subir todos los videos disponibles y seleccionados.
		\item Estudiar las principales redes para subir fotografías, seleccionar una y crear ahi la galería de imágenes del grupo.
		\item Enlazar las redes sociales con el nuevo sitio.
    \end{itemize}
	\item[Implementación de nuevos módulos:] Definir, priorizar y desarrollar los nuevos módulos que tendrá el sitio del Club:	
    \begin{itemize}
		\item Control de acceso.
		\item Gestión de participantes e integrantes.
		\item Gestión de actividades.
		\item Gestión de contenidos y materiales didácticos.
		\item Gestión de participaciones.
		\item Gestión de Asignación de Material del Club: herramientas, piezas eléctricas, robots, etc.
    \end{itemize}
\end{description}

	La última etapa deberá ser discutida para indicar que se deberá atacar en cada semestre.