\documentclass[10pt]{book}
\usepackage{cdt/cdtAnalisis}
\usepackage{tikz}
%%%%%%%%%%%%%%%%%%%%%%%%%%%%%%%%%%%%%%%%%%%%%%%%%%%%%%%%%%%%%%%%
% Datos del proyecto

\cdtOrganizacion[CNSNS--SENER]{Comisión Nacional de Seguridad Nuclear y Salvaguardias, SENER}

\cdtAutor{Coordinación de Desarrollo Tecnológico, IPN}

\cdtSistema[REPO]{Subsistema de Repositorio de Información}

\cdtProyecto[234412, IPN-23.13-SCOR2]{Sistema de Control Radiológico Versión 2.0.}

\cdtDocumento{Propuesta}{Propuesta técnica}{\DRAFT{\today}} %\RELEASE{1.0}

\cdtEntregable{E1}{Entregable 1}

% Descomentar y establecer la fecha cuando se desee congelar la fecha del documento.
%\cdtFecha{12 de Abril de 2013}

%%%%%%%%%%%%%%%%%%%%%%%%%%%%%%%%%%%%%%%%%%%%%%%%%%%%%%%%%%%%%%%%

\begin{document}

%=========================================================
% Portada
\thispagestyle{empty}

\maketitle

%=========================================================
% Indices del documento
\frontmatter
\tableofcontents
\listoffigures
\listoftables
\mainmatter
% Para esconder la información del documentador se descomenta el \hideControlVersion
%\hideControlVersion

%=========================================================
\chapter{Introducción}

\TODO[Escribir un párrafo introductorio]

%---------------------------------------------------------
\section{A quien va dirigido}

\TODO[Describir al equipo de la CDT y los perfiles a quienes va dirigido]

%---------------------------------------------------------
\section{Uso y privacidad}

\TODO[Escribir el uso que le podrán dar y las consideraciones de privacidad y derechos de autor y uso de la biblioteca.]

%---------------------------------------------------------
\section{Organización del documento}

\TODO[Indicar que sí contiene el documento, que no y el contenido de cada capítulo]

%---------------------------------------------------------
\section{Notación utilizada}

\TODO[Indicar todas las notaciones utilizadas, incluyendo estructuras y tipografía.]


%=========================================================
\chapter{El paquete {\tt cdtTheme}}

\TODO[describir el propósito del paquete, consideraciones de uso e introducir las secciones del capítulo.]

\begin{objetivos}[Durante este capítulo usted aprenderá sobre los siguientes aspectos:]
	\item \TODO[describir cada uno de los aspectos que cubre el presente capítulo.]
\end{objetivos}

\cfinput{cdt-theme.tex}

%=========================================================
\chapter{El paquete {\tt cdtBook}}

\TODO[Describir el propósito del paquete, consideraciones de uso e introducir las secciones del capítulo.]

\begin{objetivos}[Durante este capítulo usted aprenderá sobre los siguientes aspectos:]
	\item \TODO[Describir cada uno de los aspectos que cubre el presente capítulo.]
\end{objetivos}

\cfinput{cdt-book.tex}

%=========================================================
\chapter{El paquete {\tt cdtAnalisis}}
\TODO[Describir el propósito del paquete, consideraciones de uso e introducir las secciones del capítulo.]

\begin{objetivos}[Durante este capítulo usted aprenderá sobre los siguientes aspectos:]
	\item \TODO[Describir cada uno de los aspectos que cubre el presente capítulo.]
\end{objetivos}

%\cfinput{cdt-analisis.tex}

%=========================================================
\chapter{}

%=========================================================
\chapter{Introducción}


%=========================================================
\chapter{Introducción}


%==========================================================
\clossing
\end{document}
