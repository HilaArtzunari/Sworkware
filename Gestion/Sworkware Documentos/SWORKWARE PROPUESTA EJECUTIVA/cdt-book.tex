%%%%%%%%%%%%%%%%%%%%%%%%%%%%%%%%%%%%%%%%%%%%%%%%%%%%%%%%%%%%%%
%  Documento raíz: manual.tex
%  Contenido: Capítulo ``El paquete cdtBook''
%  Responsable: Ulises Vélez Saldaña
%%%%%%%%%%%%%%%%%%%%%%%%%%%%%%%%%%%%%%%%%%%%%%%%%%%%%%%%%%%%%%

%---------------------------------------------------------
\section{Paquetes que incluye}

\begin{Cdescription}
	\item[Paquete:] {\tt cdt/cdtTheme}
	\item[Propósito:] Que el documento tenga la temática de la CDT.
	\item[Configuración:] No aplica.
	\item[Consideraciones adicionales:] Revisar el capítulo anterior.
\end{Cdescription}

\begin{Cdescription}
	\item[Paquete:] {\tt longtable}%Descripción redactada por EVELYN.
	\item[Propósito:] Que se permita presentar en el documento la continuidad de una tabla muy la larga.
	\item[Configuración:] 
	\item[Consideraciones adicionales:] Revisar el manual del paquete en \url{http://tug.org}.
\end{Cdescription}

\begin{Cdescription}
	\item[Paquete:] {\tt colortbl}%Descripción redactada por EVELYN.
	\item[Propósito:] Que filas, columna o celdas de un tabla tengan color.
	\item[Configuración:] 
	\item[Consideraciones adicionales:] Revisar el manual del paquete en \url{http://tug.org}.
\end{Cdescription}

\begin{Cdescription}
	\item[Paquete:] {\tt multirow}%Descripción redactada por EVELYN.
	\item[Propósito:] Que una tabla presente múltiples renglones.
	\item[Configuración:] 
	\item[Consideraciones adicionales:] Revisar el manual del paquete en \url{http://tug.org}.
\end{Cdescription}


%---------------------------------------------------------
\section{Variables definidas}

\begin{Cdescription}
	\item[Variable:] {\tt h}
	\item[Tipo:] Tipo de columna para tablas {\tt multirow}.
	\item[Propósito:]
	\item[Uso:]
	\item[Ejemplo:]
\end{Cdescription}

\begin{Cdescription}
	\item[Variable:] {\tt $\backslash$varSistema}
	\item[Tipo:] Comando de \LaTeX.
	\item[Propósito:]
	\item[Uso:]
	\item[Ejemplo:]
\end{Cdescription}

\begin{Cdescription}
	\item[Variable:] {\tt $\backslash$varTitle}
	\item[Tipo:] Comando de \LaTeX.
	\item[Propósito:]
	\item[Uso:]
	\item[Ejemplo:]
\end{Cdescription}

\begin{Cdescription}
	\item[Variable:] {\tt $\backslash$varSubTitle}
	\item[Tipo:] Comando de \LaTeX.
	\item[Propósito:]
	\item[Uso:]
	\item[Ejemplo:]
\end{Cdescription}

\begin{Cdescription}
	\item[Variable:] {\tt $\backslash$varOrganizacion}
	\item[Tipo:] Comando de \LaTeX.
	\item[Propósito:]
	\item[Uso:]
	\item[Ejemplo:]
\end{Cdescription}

\begin{Cdescription}
	\item[Variable:] {\tt $\backslash$varAuthor}
	\item[Tipo:] Comando de \LaTeX.
	\item[Propósito:]
	\item[Uso:]
	\item[Ejemplo:]
\end{Cdescription}

\begin{Cdescription}
	\item[Variable:] {\tt $\backslash$varProyecto}
	\item[Tipo:] Comando de \LaTeX.
	\item[Propósito:]
	\item[Uso:]
	\item[Ejemplo:]
\end{Cdescription}

\begin{Cdescription}
	\item[Variable:] {\tt $\backslash$varCveProyecto}
	\item[Tipo:] Comando de \LaTeX.
	\item[Propósito:]
	\item[Uso:]
	\item[Ejemplo:]
\end{Cdescription}

\begin{Cdescription}
	\item[Variable:] {\tt $\backslash$varEtapaProy}
	\item[Tipo:] Comando de \LaTeX.
	\item[Propósito:]
	\item[Uso:]
	\item[Ejemplo:]
\end{Cdescription}

\begin{Cdescription}
	\item[Variable:] {\tt $\backslash$varUsoProy}
	\item[Tipo:] Comando de \LaTeX.
	\item[Propósito:]
	\item[Uso:]
	\item[Ejemplo:]
\end{Cdescription}

\begin{Cdescription}
	\item[Variable:] {\tt $\backslash$varCveDocumento}
	\item[Tipo:] Comando de \LaTeX.
	\item[Propósito:]
	\item[Uso:]
	\item[Ejemplo:]
\end{Cdescription}

\begin{Cdescription}
	\item[Variable:] {\tt $\backslash$varDocumento}
	\item[Tipo:] Comando de \LaTeX.
	\item[Propósito:]
	\item[Uso:]
	\item[Ejemplo:]
\end{Cdescription}

\begin{Cdescription}
	\item[Variable:] {\tt $\backslash$varDocVersion}
	\item[Tipo:] Comando de \LaTeX.
	\item[Propósito:]
	\item[Uso:]
	\item[Ejemplo:]
\end{Cdescription}

\begin{Cdescription}
	\item[Variable:] {\tt $\backslash$varFecha}
	\item[Tipo:] Comando de \LaTeX.
	\item[Propósito:]
	\item[Uso:]
	\item[Ejemplo:]
\end{Cdescription}


%---------------------------------------------------------
\section{Comandos}

\begin{Cdescription}
	\item[Comando:] {\tt $\backslash$sistema}
	\item[Propósito:] Definir el nombre del sistema relacionado con el proyecto.
	\item[Ejemplo:] $\backslash$sistema[SEP]\{Secretaría de Educación Pública\}
	
\end{Cdescription}

\begin{Cdescription}
	\item[Comando:] {\tt $\backslash$title}
	\item[Propósito:] Definir el título del documento.
	\item[Ejemplo:] $\backslash$title\{La calidad y la productividad en las empresas de servicios\}
\end{Cdescription}

\begin{Cdescription}
	\item[Comando:] {\tt $\backslash$subtitle}
	\item[Propósito:] Definir el subtitulo del documento.
	\item[Ejemplo:]$\backslash$subtitle\{Calidad y productividad\}
\end{Cdescription}

\begin{Cdescription}
	\item[Comando:] {\tt $\backslash$organizacion}
	\item[Propósito:] Definir la organización a quien va dirigido el documento.
	\item[Ejemplo:] $\backslash$organizacion[DGEI--SEP]\{Dirección General de Educación Indígena, SEP\}
\end{Cdescription}

\begin{Cdescription}
	\item[Comando:] {\tt $\backslash$author}
	\item[Propósito:] Definir el autor del documento
	\item[Ejemplo:]$\backslash$author\{Dr. López Costa Juan José\}
\end{Cdescription}

\begin{Cdescription}
	\item[Comando:] {\tt $\backslash$fecha}
	\item[Propósito:] Definir la fecha del documento
	\item[Ejemplo:]$\backslash$Martes 24 de Octubre, 2016.\}
\end{Cdescription}


\begin{Cdescription}
	\item[Comando:] {\tt $\backslash$proyecto}
	\item[Propósito:] Definir la clave y el nombre del proyecto del documento
	\item[Ejemplo:] $\backslash$proyecto[CVE]\{Reingeniería de procesos\}
\end{Cdescription}

\begin{Cdescription}
	\item[Comando:] {\tt $\backslash$documento}
	\item[Propósito:] Establecer la clave, nombre y versión del documento, incluyendo etapa del proyecto y uso del documento.
	\item[Ejemplo:] $\backslash$documento[C1-DT] \{Componente 1: Documentación Técnica\}
\end{Cdescription}
%
\begin{Cdescription}
	\item[Comando:] {\tt $\backslash$makeDocInfo}
	\item[Propósito:] Generar la ficha de información del documento
 Ejemplo: Generalmente se usan combinaciones de la siguiente secuencia
	\item[Ejemplo:] Use la siguiente secuencia para generar una ficha completa de revisión:
	\begin{Citemize}
	 	\item {\tt $\backslash$makeDocInfo}
	 	\item {\tt $\backslash$makeElemRefs}
	 	\item {\tt $\backslash$makeDocRefs}
	 	\item {\tt $\backslash$makeObservaciones}
	 	\item {\tt $\backslash$makeFirmas}
	\end{Citemize}
\end{Cdescription}

\begin{Cdescription}
	\item[Comando:] {\tt $\backslash$elemRefs}, {\tt $\backslash$elemItem}
	\item[Propósito:] Listar los documentos relacionados con el documento actual. Este comando permite definitivos, para generar la tabla de entradas use {\tt $\backslash$makeElemRefs}.
	\item[Ejemplo:]use:\\
	{\tt $\backslash$elemRefs\{\\
               \hspace*{1cm}$\backslash$elemItem\{C2-GP\}\{1.0\}\{Guión de pruebas\}\\
               \hspace*{1cm}$\backslash$elemItem\{C3-BP\}\{1.2\}\{Base de pruebas\}\\
               \hspace*{1cm}$\backslash$elemItem\{C4-BD\}\{2.9\}\{Base de datos\}\\
               \hspace*{1cm}...\\
           \}}
\end{Cdescription}
%
\begin{Cdescription}
	\item[Comando:] {\tt $\backslash$makeElemRefs}
	\item[Propósito:] Expande una tabla con la definición de los documentos definidos previamente mediante {\tt $\backslash$elemRefs}.
	\item[Ejemplo:] {\tt $\backslash$makeElemRefs}
\end{Cdescription}
%
%\begin{Cdescription}
%	\item[Comando:] {\tt $\backslash$}
%	\item[Propósito:] 
%	\item[Ejemplo:]
%\end{Cdescription}
%
%\begin{Cdescription}
%	\item[Comando:] {\tt $\backslash$}
%	\item[Propósito:]
%	\item[Ejemplo:]
%\end{Cdescription}
%
%\begin{Cdescription}
%	\item[Comando:] {\tt $\backslash$}
%	\item[Propósito:]
%	\item[Ejemplo:]
%\end{Cdescription}
%
%\begin{Cdescription}
%	\item[Comando:] {\tt $\backslash$}
%	\item[Propósito:]
%	\item[Ejemplo:]
%\end{Cdescription}
%
%\begin{Cdescription}
%	\item[Comando:] {\tt $\backslash$}
%	\item[Propósito:]
%	\item[Ejemplo:]
%\end{Cdescription}



%---------------------------------------------------------
\section{}

%---------------------------------------------------------
\section{}

%---------------------------------------------------------
\section{}

