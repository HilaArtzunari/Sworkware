% \IUref{IUAdmPS}{Administrar Planta de Selección}
% \IUref{IUModPS}{Modificar Planta de Selección}
% \IUref{IUEliPS}{Eliminar Planta de Selección}

% 


% Copie este bloque por cada caso de uso:
%-------------------------------------- COMIENZA descripción del caso de uso.

%\begin{UseCase}[archivo de imágen]{UCX}{Nombre del Caso de uso}{
\begin{UseCase}{CUE1}{Login en plataforma de escritorio}{
El Médico, el recepcionista, el cajero y el administrador podrán acceder a su cuenta de usuario desde una ventana de inicio de sesión UI01Escritorio mediante el ingreso de su correo y su contraseña.\\
}
\UCitem{Versión}{1}
\UCitem{Actor}{Medico, recepcionista, cajero ó administrador}
\UCitem{Propósito}{Acceder a la vista correspondiente}
\UCitem{Entradas}{
- Correo: Cadena de caracteres con el formato: ejemplo@dominio.com
- Contraseña: Cadena de caracteres de mínimo 6 y máximo 20.
}
\UCitem{origen}{Todas las entradas del teclado}
\UCitem{Salidas}{
- Mensaje de confirmación: “Hola nombre de usuario”
}
\UCitem{Destino}{UI02 Medico, UI03recepcionista, UI04cajero, UI05administrador}
\UCitem{Precondiciones}{
- Debe existir el idcorreo de la cuenta de usuario del correo ingresado
- La contraseña ingresada debe coincidir con la contraseña asignada a la cuenta de usuario de idcorreo
- Conexión a la base de datos.
}
\UCitem{Postcondiciones}{
- Se inicia la sesión.

- Se llenan los datos de una estructura Usuario de la clase Estructuras.
- Se pasa por referencia la estructura usuario a la UI correspondiente al tipo de usuario(UI02Medico, UI03Recepcionista, UI04Cajero, UI05Administrador).
- Se cierra la UI01Escritorio.
- Se muestra la UI correspondiente al tipo de usuario(UI02 Medico, UI03Recepcionista, UI04Cajero, UI05Administrador).
}
\UCitem{Errores:}{
- Si el correo ingresado no existe en la tabla de usuario de la base de datos, se informa al usuario mediante un mensaje que el usuario ingresado no existe
- Si la contraseña ingresada no coincide con la del idcorreo ingresado se informa al usuario mediante un mensaje que la contraseña es incorrecta.
- Si el usuario no ingresa nada, no se realiza ninguna acción y se notifica que el usuario ingresado no existe
- Si inicia secion un usuario tipo web, se notifica que no tiene derechos para acceder en esta plataforma

- Si inicia secion un usuario tipo desconocido, se notifica del error.
}
\UCitem{Tipo de ejecución}{Primario}
\UCitem{Autor}{Ulises Pérez Montiel}
\UCitem{Revisó}{Gabriela Moreno González.}
\end{UseCase}
\hspace{1cm}

\begin{UCtrayectoria}{Principal}
\UCpaso[\UCactor] Ejecuta SAC.jar 
\UCpaso Muestra la vista UI01Escritorio
\UCpaso[\UCactor] Ingresa su correo y su contraseña en las cajas de texto correspondientes de la UI01Escritorio.
\UCpaso[\UCactor] Oprime el botón de 'Ingresar.'
\UCpaso Valida que el usuario ingrese algo en las dos cajas de texto \Trayref{A}
\UCpaso Ejecuta un query donde consulta a la base de datos el idcorreo de la tabla usuario donde idcorreo sea igual al correo ingresado\Trayref{B}
\UCpaso Ingresa a la base de datos y esta devuelve un Resultset con el idcorreo ingresado
\UCpaso Ejecuta un query donde consulta a la base datos el password de la tabla usuario donde idcorreo sea igual al correo ingresado
\UCpaso Valida la contraseña ingresada comparándola con la contraseña real consultada en la base de datos \Trayref{C}
\UCpaso Ejecuta un query donde consulta a la base de datos el tipo de la tabla usuario donde idcorreo sea igual al correo ingresado
\UCpaso Valida si el tipo de usuario es tipo 0 (usuario web)\Trayref{D}
\UCpaso Valida si el tipo de usuario es tipo 1 (cajero)\Trayref{E}
\UCpaso Valida si el tipo de usuario es tipo 2 (Recepcionista)\Trayref{F}
\UCpaso Valida si el tipo de usuario es tipo 3 (Medico)\Trayref{G}
\UCpaso Valida si el tipo de usuario es tipo 4 (Administrador)\Trayref{H}
\UCpaso Valida si el tipo de usuario es tipo mayor que 4 (desconocido)\Trayref{I}

\end{UCtrayectoria}

\begin{UCtrayectoriaA}{A}{El usuario no ingresó el Correo ó la contraseña}
\UCpaso Muestra un mensaje {\bf MSG1-}``El usuario [{\em Correo ingresado}] No existe.''.
\UCpaso [\UCactor] Oprime el botón \IUbutton{Aceptar}. .
\UCpaso Continua en el paso 2 del \UCref{CUE1}.
\UCpaso[] Termina el caso de uso.
\end{UCtrayectoriaA}

\begin{UCtrayectoriaA}{B}{La base de datos devuelve un error indicando que no existe el idcorreo ingresado}
\UCpaso Muestra un mensaje '{\bf MSG2-}``El usuario [{\em Correo ingresado}] No existe.''.
\UCpaso[\UCactor] oprime el botón \IUbutton{Aceptar}.
\UCpaso Continua en el paso 2 del \UCref{CUE1}
\UCpaso[] Termina el caso de uso.
\end{UCtrayectoriaA}

\begin{UCtrayectoriaA}{C}{El sistema detecta que el password ingresado no es correcto}
\UCpaso Muestra un mensaje{\bf MSG3-}``La contraseña del usuario[{\em Correo ingresado}] Es incorrecta''.
\UCpaso[\UCactor] Oprime el botón \IUbutton{Aceptar}. 
\UCpaso Continua en el paso 2 del \UCref{CUE1}.
\UCpaso[] Termina el caso de uso.
\end{UCtrayectoriaA}

\begin{UCtrayectoriaA}{D}{Un usuario web intenta acceder a la plataforma de escritorio}
\UCpaso Muestra un mensaje {\bf MSG4-}``Los usuarios web no pueden acceder en esta plataforma '' 
\UCpaso Continua en el paso 2 del \UCref{CUE1}.
\UCpaso[] Termina el caso de uso.
\end{UCtrayectoriaA}

\begin{UCtrayectoriaA}{E}{El usuario es un Cajero}
\UCpaso Ejecuta un query donde cónsula a la base de datos toda la información de usuario del cajero
\UCpaso El sistema muestra la UI04Cajero
\UCpaso[] Termina el caso de uso.
\end{UCtrayectoriaA}

\begin{UCtrayectoriaA}{F}{El usuario es un Recepcionista}
\UCpaso Ejecuta un query donde cónsula a la base de datos toda la información de usuario del Recepcionista
\UCpaso El sistema muestra la UI03Recepcionista
\UCpaso[] Termina el caso de uso.
\end{UCtrayectoriaA}

\begin{UCtrayectoriaA}{G}{El usuario es un Medico}
\UCpaso Ejecuta un query donde cónsula a la base de datos toda la información de usuario del Medico
\UCpaso El sistema muestra la UI02Medico
\UCpaso[] Termina el caso de uso.
\end{UCtrayectoriaA}

\begin{UCtrayectoriaA}{H}{El usuario es un Administrador}
\UCpaso Ejecuta un query donde cónsula a la base de datos toda la información de usuario del Adminsitrador
\UCpaso El sistema muestra la UI05Administrador
\UCpaso[] Termina el caso de uso.
\end{UCtrayectoriaA}

\begin{UCtrayectoriaA}{I}{Un usuario de tipo desconocido intenta acceder a la plataforma}
\UCpaso El sistema muestra un mensaje de error {\bf MSG5-}``El usuario [{\em Correo ingresado}] No deberia existir, consulte a su proveedor de software que pudo haber sucedido.''.
\UCpaso[\UCactor] Oprime el botón \IUbutton{Aceptar}. 
\UCpaso Continua en el paso 2 del \UCref{CUE1}. 
\UCpaso[] Termina el caso de uso.
\end{UCtrayectoriaA}
%-------------------------------------- TERMINA descripción del caso de uso.

