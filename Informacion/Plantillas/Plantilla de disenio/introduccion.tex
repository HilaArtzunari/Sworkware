
En este documento se presentará el análisis, diseño, construcción y las pruebas de nuestro Sistema Administrador de Clínicas, así como la metodología usada y las etapas que harán que el sistema vaya creciendo y permita resolver las problemáticas que actualmente enfrenta la empresa.

El presente documento se encuentra divido por 7 bloques: Introducción, Análisis de problema, Propuesta de solución, Modelo de Negocios, Modelo de despliegue del sistema, Modelo de comportamiento y modelo de la iteración.

En la introducción que en este momento lee se da una pequeña reseña del trabajo, así como los acrónimos, abreviaturas y referencias bibliográficas que se han consultado con el fin de diseñar este documento de la forma más amigable y entendible para el lector.
En la sección de Análisis del problema se tratará el contexto del sistema, el cual define cómo se mueve la empresa actualmente; los procesos actuales, que describe quiénes y cuál es su puesto dentro del sistema; los problema identificados, que son las razones que lleva a la empresa a solicitar un sistema y que no permiten que funcione de manera eficaz; y, finalmente, las propuestas de solución, que son las alternativas que se podrían aplicar para resolver los problemas identificados, de esas alternativas de solución se seleccionará la que mejor resuelva el problema y cumpla los requisitos que la empresa solicita.

La propuesta de solución se desglosará planteando primeramente los objetivos: un objetivo general y varios partículares, que serán las metas que queremos lograr con nuestro sistema; y se tratará el modelo de despliegue abarcando los requerimientos no funcionales, el modelado de dicho despliegue y las especificaciones de la plataforma.

El modelo de negocios nos permitirá definir cómo trabaja la empresa actualmente y si en algún punto el software que se planea implementar podría llegar a cambiar la forma en la que se mueve la empresa. Primeramente se creará un glosario de términos para entender la jerga de los empleados, se tratarán los procesos ajustados, así como los procesos actuales, la descripción de atributos y finalmente las reglas del negocio, que son las principales y son las que rigen todo sistema.

El modelo de despliegue del sistema en una sección encargada de mostrar cómo nuestro sistema se va desarrollando a través del tiempo y cómo está estructurado.

El modelo de comportamiento describe qué funcionalidades tiene el sistema y cómo debe reaccionar ante diversos eventos que genere el usuario, describiendo sus atributos y cómo se va a mover el sistema en caso de entradas no esperadas.

El modelo de iteraciones presentará la descripción completa de la sinterfaces de usuario y cómo éste puede manipularlas e interactuar con ellas para obtener un resultado definido.

Este documento va dirigido al profesor Ulises Vélez Saldaña, profesor de la Escuela Superior de Cómputo del Instituto Politécnico Nacional como un proyecto de desarrollo de software.

Este documento será realizado por 'The Dream Team', conformado por:

- Martínez Vilchis Juan Moisés.

- Moreno González Gabriela.

- Pérez Montiel Ulises.

- Reynoso Rodríguez Erick Rubén.

Realizado en la Escuela Superior de Cómputo del Instituto Politécnico Nacional, mediante una organización secuencial de las partes que se irán cubriendo del documento y de la presentación final.

%--------------------------------------------------
\section{Alcance}
Nuestro proyecto planea cubrir todos los requerimientos funcionales y no funcionales que serán planteados y analizados a lo largo del proyecto para resolver las problemáticas principales. Se espera que los problemas secundarios sean tratados en tiempo posterior a la entrega de este proyecto.

%--------------------------------------------------
\section{Notaciones, estándares y UML.}

\begin{description}
	\item[Servidor:] Como el manejo será local, un servidor se entiende como el software que configura un PC como servidor para facilitar el acceso a la red y sus recursos.
\end{description}

\begin{description}
	\item[UML:] Unified Modeling Language (Lenguaje de Modelado Unificado), es un lenguaje estándar utilizado para modelar diagramas de clases, de secuencias, etc.
\end{description}

\begin{description}
	\item[Scrum:] Es una metodología de desarrollo de software.
\end{description}

\begin{description}
	\item[Diagrama de Ishiwaka:] Son diagramas empleados para profundizar de una manera gráfica al menos 6 aspectos dentro de una problemática global.
\end{description}

\begin{description}
	\item[Requerimientos funcionales:] Son todas aquellas acciones que requiere hacer el sistema y que necesita el usuario.
\end{description}

\begin{description}
	\item[requerimientos no funcionales:] Son todas aquellas acciones o aspectos que deben cumplir las acciones que se realizan 'detrás' de los requerimientos funcionales para su correcta implementación.
\end{description}

\begin{description}
	\item[Business Motivation Model:] Provee un esquema o estructura para desarrollar, comunicar, y gestionar los planes de negocio de una manera organizada.
\end{description}

\begin{description}
	\item[Business Process Modeling Notation:] Es una notación utlizada para modelar procesos dentro de una empresa mediante el conocimiento de las reglas del negocio y de los procesos actuales.
\end{description}

\begin{description}
	\item[Caso de uso:] Es una descriptiva de una acción que debe realizar el sistema, especificando los valores de entrada, las salidas y las pantallas en las que se llevará a cabo dicha acción.
\end{description}

\begin{description}
	\item[Botón:] Es un componente de java swing que permite la realización de determinadas acciones al presionarse.
\end{description}

\begin{description}
	\item[Campo de texto:] Es un componente de java swing que permite al usuario ingresar una cadena de texto.
\end{description}

\begin{description}
	\item[Reglas del negocio:] Son todas aquellas sentencias que definen la operación del negocio y permiten a los participantes tomar decisiones.
\end{description}

\begin{description}
	\item[Package:] Es una agrupación de clases afines.
\end{description}

\begin{description}
	\item[Diagrama de despliegue:] Son los diagramas que muestran las clases que contiene un package.
\end{description}