En este capítulo se analizarán los problemas que tiene la manera en la que se maneja actualmente el proceso de compra y venta de material electrónico y el por qué es necesario el sistema para agilizar determinados procesos, así como las alternativas de solución que se proponen y cuál de éstas es la que mejor resuelve la problemática. \\

También se tratarán los actores dentro del sistema actual y cómo se manejarán dentro de nuestro sistema. Se describe la problemática y los problemas puntuales encontrados, desglosados por el diagrama de Ishikawa para profundizarlos y resolverlos.

%--------------------------------------------------
\section{Contexto del sistema}

En el Instituto Politécnico Nacional se imparten diversas unidades de aprendizaje de acuerdo a la escuela y a la carrera que se esté cursando. Algunas de ellas pertenecen al área de Electrónica como Análisis de Circuitos e Instrumentación, por lo que los estudiantes requieren adquirir los componentes electrónicos para realizar sus prácticas, sin embargo suele ser costoso o difícil de haya algunos de éstos.

Los problemas más comunes al ir a comprar un componente electrónico son:
\begin{enumerate}
	\item Ninguna tienda tiene el componente
	\item El componente ha dejado de ser fabricado
	\item El componente es muy caro
	\item El componente no tiene especificaciones
	\item El componente viene dañado y no posee garantía
\end{enumerate}

%--------------------------------------------------
\section{Procesos actuales}

Los procesos actuales son:

\begin{enumerate}
	\item Difundir la venta de un material electrónico: La mayoría de los estudiantes optan por publicar la venta ya sea en Facebook o bien en Mercado Libre, otros lo difunden con sus amigos y conocidos para que ellos los apoyen a encontrar posibles interesados.
	\item Vender o comprar un material electrónico: Los estudiantes suelen acordar un punto de reunión ya sea en algún campus de su escuela o bien de forma externa donde intercambian el material por el pago acordado, otros que ofertaron en lugares como Mercado Libre pueden pedir el pago por medio de un depósito bancario pero esto es muy inusual.
	\item Encontrar información sobre material electrónico que está en venta: Esto resulta casi imposible, a menos que veas la publicación en Facebook (en caso de ser reciente) o el vendedor o alguno de sus conocidos te haga saber la oferta.
	\item Calificar al vendedor y comprador: La única forma de hacer esto es que le hables o no a tus conocidos sobre la persona que te vendió o compró el material. En caso de haber usado Mercado Libre, esta plataforma ya cuenta con una sección para calificar una venta.
	\item Reportar falsa publicidad: Es imposible, si compras el material electrónico y al probarlo no sirve, es muy complicado que el vendedor se haga responsable. En caso de ser Mercado Libre, puedes reportarlo.
	\item Cancelar una compra o venta: En caso de ser acordado vía palabra, puedes cancelarlo del mismo modo, en caso de haber usado Mercado Libre, no permite que se cancele la venta o compra, solo te calificarán con un negativo.
	\item Modificar un material electrónico en venta: Si la publicación está en Facebook o Mercado Libre, puedes editarla, si usaste la difusión por medio de tus contactos, el problema viene con que no sabes hasta donde ha llegado dicha información.
	\item Eliminar una oferta de material electrónico en venta: Si la publicación está en Facebook o Mercado Libre, puedes borrarla, si lo difundiste con tus conocidos difícilmente difundirán que ya no lo estás vendiendo.

	\item Conocer si alguien está interesado en tu material electrónico en venta: Esto es, o te comentan la publicación en Facebook o Mercado Libre, o algunos de tus conocidos te dice que encontró a un interesado, que puede darse el caso de que se le olvide avisarte.
	\item Donar material electrónico: Esto resulta casi imposible, a menos que sea un conocido el que te pida prestado un material y elijas regalarlo, de otra manera el llegar con alguien y decirle si necesita material y que se lo puedes regalar resulta hasta incómodo.
\end{enumerate}

% - - - - - - - - - - - - - - - - - - - - - - - - -
\subsection{Participantes}

Nombre: Estudiante.

Descripción: Es la persona interesada en comprar o vender material electrónico.

Responsabilidades: Es el encargado de publicar su material electrónico en venta, comprar en caso de requerirlo y mantener el contacto con el comprador o vendedor, así como su punto de reunión, verificar en el sistema que la compra o venta se hizo correctamente y calificar al otro estudiante involucrado. \\

Nombre: Administrador.

Descripción: Es la persona encargada de administrar los perfiles de los estudiantes, así como del material electrónico que se encuentra ofertado.

Responsabilidades: Bloquear o eliminar perfiles de usuarios, verificar que el contenido subido al sitio sea el adecuado y responder dudas acerca del propósito y los procesos realizados dentro del sitio.

% - - - - - - - - - - - - - - - - - - - - - - - - -
\subsection{Procesos}

diagramas y explicación de los procesos: describir las actividades.

%--------------------------------------------------
\section{Problemas identificados}

% - - - - - - - - - - - - - - - - - - - - - - - - -
\subsection{Problema general}

Los estudiantes requieren vender, comprar o donar material electrónico, sin embargo no existe una plataforma que les permita realizar esto de forma rápida y sencilla.

% - - - - - - - - - - - - - - - - - - - - - - - - -
\subsection{Descomposición del problema}

\begin{enumerate}
	\item Componentes que no se vuelven a usar: Muchas veces los estudiantes realizan la inversión para comprar sus materiales y nunca los vuelven a usar ni en el resto de su carrera ni cuando comienzan a laborar.

	\item Ineficiencia al localizar un componente:Los componentes comprados hace años por otros estudiantes o jamás se ponen en venta o simplemente los publican en sitios poco convenientes como Facebook y la publicación jamás es vista por el público que está interesado.

	\item Inexistencia de la donación de componentes: La única forma en la que algunos estudiantes regalan sus componentes es cuando conocen a la persona y le solicita que se los preste, a lo que el estudiante prefiere regalarlos porque no los va a volver a usar.
\end{enumerate}

% - - - - - - - - - - - - - - - - - - - - - - - - -
\subsection{Análisis de causas}

Las principales causas por las cuales la venta o donación se vuelve compleja son:

\begin{enumerate}
	\item Los estudiantes no poseen una herramienta que les facilite la interacción entre el comprador y el vendedor para adquirir sus materiales.
	\item Los estudiantes tienen miedo de comprar material que no funcione o que no sea el que le solicitaron sus profesores.
	\item Los estudiantes quieren conseguir material barato, sin embargo los materiales que venden en tiendas oficiales o en el centro normalmente es nuevo y resulta caro.
	\item Los estudiantes quieren mejor comprar todo el material de una práctica completa que estar buscando en todos lados componente por componente, sin embargo muy rara vez alguien lo vende así.
	\item No existe un medio de comunicación oficial para acordar compras.
	\item No existe seguridad en las compras de los componentes.
\end{enumerate}
%--------------------------------------------------
\section{Propuesta de solución}

% - - - - - - - - - - - - - - - - - - - - - - - - -
\subsection{Alternativas de solución}

Las alternativas de solución disponibles son:

\begin{enumerate}
	\item Solicitar a cada escuela que aparten un salón para almacenar material electrónico y se haga un sistema para préstamos y devoluciones: Esta solución sería la más eficiente y la que evitaría que los estudiantes gastaran tanto en su material, ya que podrías pedirle al instituto que te preste el material para tu práctica con el compromiso de que lo vas a devolver en un determinado período y de forma funcional, el problema surge al pedir el salón donde se almacenará todo el material y quienes lo van a administrar y si se les pagará o simplemente serán de servicio social.
	\item Crear un sistema que permite a los estudiantes comprar material electrónico desde un sitio web: Esta solución es de las más fáciles, ya que la interacción se haría desde el sitio web y no se altera el organigrama d ela institución ni se requieren permisos mayores.
\end{enumerate}
