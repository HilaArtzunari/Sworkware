<<<<<<< HEAD
\begin{UseCase}{CU12}{Eliminar a usuaio.}{
		Permite a un administrador eliminar a un usuario
	}
		\UCitem{Versión}{1.0}
		\UCitem{Actor}{Administrador.}
		\UCitem{Propósito}{Eliminar a usuarios que cometan actos fraudolentos y/o ofensivos.}
		\UCitem{Entradas}{clic de mouse.}
		\UCitem{Origen}{Mouse.}
		\UCitem{Salidas}{Mensaje con la leyenda "Esta seguro de eliminar a este usuario?".}
		\UCitem{Destino}{Pantalla.}
		\UCitem{Precondiciones}{Tener el rol de administrador del sitio e ingresar al sitio con cuenta de administrador.}
		\UCitem{Postcondiciones}{Un cambio en el registro de usuarios en el sitio.}
		\UCitem{Errores}{{\bf E1:} ``Error de conexión al servidor.'' -- El sistema muestra el Mensaje {\bf MSG1-}``Tú solicitud no puede ser enviado en este momento. Inténtalo más tarde.'' y continua al paso 1.}
		
		\UCitem{Tipo}{Caso de uso primario}
		\UCitem{Observaciones}{}
		\UCitem{Autor}{Angel Ferreira Osorno.}
		\UCitem{Revisor}{Diana Laura Mejía Mendoza.}
	\end{UseCase}

	\begin{UCtrayectoria}{Principal} 
		\UCpaso[\UCactor] Selecciona el botón \IUbutton{Eliminar} en la \IUref{IU12}{Eliminar perfil} \Trayref{B}
		\UCpaso abre una ventana con la leyenda "Esta seguro de eliminar a este usuario?".
		\UCpaso[\UCactor] Selecciona dando click al botón \IUbutton{Si} \Trayref{A}
		\UCpaso abre una ventana con la leyenda "Eliminado exitosamente" \Trayref{B}.
		\UCpaso elimina el registro en el sitio del usuario.
	\end{UCtrayectoria}

		\begin{UCtrayectoriaA}{A}{Cancelar.}
			\UCpaso[\UCactor]da clic al boton \IUbutton{Cancelar} 
			\UCpaso Continúa en el paso 1 del \UCref{CU12}.
		\end{UCtrayectoriaA}
		
		\begin{UCtrayectoriaA}{B}{Falla en la conexión con el servidor.}
			\UCpaso Muestra el Mensaje {\bf MSG1-}``Tú solicitud no puede ser enviada en este momento. Inténtalo más tarde.''.
			\UCpaso Continúa en el paso 1 del \UCref{CU11}.
=======
\begin{UseCase}{CU12}{Eliminar perfil de usuario.}{
		El administrador podrá eliminar una cuenta de usuario que se encuentre bloqueada, que lleve mucho tiempo inactiva o que haya sido denunciada al menos 10 veces con motivos justificados.
	}
		\UCitem{Versión}{1.0}
		\UCitem{Actor}{Administrador}
		\UCitem{Propósito}{Eliminar perfiles de usuario que jamás se volverán a usar o no cumplan la normatividad del sitio}
		\UCitem{Entradas}{Correo del usuario a eliminarse: cadena de caracteres con el formato example@domain.com}
		\UCitem{Origen}{Teclado}
		\UCitem{Salidas}{Mensaje {\bf MSG30-}``El usuario user\_name ha sido eliminado satisfactoriamente.''}
		\UCitem{Destino}{Pantalla.}
		\UCitem{Precondiciones}{- Estar registrado como administrador del sitio.
		
		- Haber iniciado sesión.
		
		- Conexión a internet.}
		\UCitem{Postcondiciones}{El usuario correspondiente al correo será eliminado de los registros.}
		\UCitem{Errores}{{\bf E1:} ``Error de conexión al servidor.'' -- El sistema muestra el Mensaje {\bf MSG1-}``No puedes eliminar al usuario en este momento. Inténtalo más tarde.'' y continua al paso 3.}
		
		\UCitem{Tipo}{Caso de uso secundario. Viene de \UCref{CU1}}
		\UCitem{Observaciones}{Ninguna}
		\UCitem{Autor}{Gabriela Moreno González.}
		\UCitem{Revisor}{Diana Laura Mejía Mendoza.}
	\end{UseCase}

	\begin{UCtrayectoria}{Principal}
		\UCpaso[\UCactor] Da clic en el botón \IUbutton{Eliminar un usuario} de la pantalla \IUref{UI18}{Ver perfil de administrador}
		\UCpaso Abre una ventana con campo de texto.
		\UCpaso[\UCactor] Escribe el correo del usuario a ser eliminado.
		\UCpaso[\UCactor] Presiona el botón \IUbutton{Eliminar}
		\UCpaso Busca al usuario con el correo ingresado y lo elimina \Trayref{A} \Trayref{B}
		\UCpaso Muestra el mensaje {\bf MSG30-}``El usuario user\_name ha sido eliminado satisfactoriamente.''
	\end{UCtrayectoria}

		\begin{UCtrayectoriaA}{A}{No hay ningún usuario con ese correo.}
			\UCpaso Muestra el mensaje {\bf MSG31-}``No existe un usuario con esa cuenta de correo.''
			\UCpaso Continua en el paso 1 de \UCref{CU12}.
		\end{UCtrayectoriaA}
		
		\begin{UCtrayectoriaA}{B}{Falla en la conexión con el servidor.}
			\UCpaso Muestra el Mensaje {\bf MSG1-}``Tú denuncia no puede ser enviada en este momento. Inténtalo más tarde.''
			\UCpaso Terminal el caso de uso.
>>>>>>> origin/master
		\end{UCtrayectoriaA}
		
		
		
%-------------------------------------- TERMINA descripción del caso de uso.