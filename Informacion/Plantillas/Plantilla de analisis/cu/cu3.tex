% \IUref{IUAdmPS}{Administrar Planta de Selección}
% \IUref{IUModPS}{Modificar Planta de Selección}
% \IUref{IUEliPS}{Eliminar Planta de Selección}

% 


% Copie este bloque por cada caso de uso:
%-------------------------------------- COMIENZA descripción del caso de uso.

%\begin{UseCase}[archivo de imágen]{UCX}{Nombre del Caso de uso}{
	\begin{UseCase}{CU3}{Agregar componente}{
		Permite al usuario poder agregar un nuevo componente para su venta o donaci\'on.
	}
		\UCitem{Versión}{0.1}
		\UCitem{Actor}{Ususario}
		\UCitem{Propósito}{Agregar nuevo componente para su venta o donaci\'on.}
		\UCitem{Resumen}{El usuario podr\'a agregar nuevos componentes en su perfil que podr\'an ser vistos por otros usuarios para su adquisici\'on.}
		
		\UCitem{Entradas}{El usuario podr\'a insertar nombre, modelo, identificador, unidad de aprendizaje, escuela del producto a ofertar el cual ser\'a caracteristico por ser una cadena de caracteres alfanumericos con tama\~no m\'aximo de 70 caracteres, imagen en formato jpg.}

		\UCitem{Origen}{Teclado.}		
		
		\UCitem{Salidas}{MSG1 ''Componente agregado con exito''.}
		
		\UCitem{Destino}{Pantalla del usuario.}

		\UCitem{Precondiciones}{El usuario debe tener un componente para agregar as\'i como la informaci\'on solicitada para describirlo.}

		\UCitem{Postcondiciones}{El sistema realizar\'n el registro al sistema.}
		
		\UCitem{Tipo}{Primario.}

		\UCitem{Observaciones}{:).}
						
				
		\UCitem{Autor}{Luis \'Angel Corona Elizalde.}

		\UCitem{Revisor}{Gaby.}
		
	\end{UseCase}		

	\begin{UCtrayectoria}{Principal}
		\UCpaso[\UCactor] El usuario ingresa a la secci\'on de agregar componente. \label{CU2alta}.
		\UCpaso[\UCactor] El usuario presiona el boton de agregar nuevo componente.%\Trayref{B}\label{CU1login} %\BRref{BR129}{Determinar si un Estudiante puede inscribir Seminario.} \Trayref{A}.
		\UCpaso Dirige al usuario al formulario para agregar el nuevo componente.% Despliega la \IUref{UI32}{Pantalla de Selección de Seminario} con la lista de Seminarios Disponibles.
		%\UCpaso Muestra el formulario%.\Trayref{A}\Trayref{B}%%Selecciona el Seminario en el que desea inscribirse \Trayref{B}\label{CU17SeleccionarSeminario}.
		\UCpaso[\UCactor] Llena los campos solicitados por el sistema.
		\UCpaso Revisa que los campos hayan sido escritos. \Trayref{A}\label{CU2altadeusuario}
	%	\UCpaso Realiza la validación de contraseña valida	\Trayref{B}\label{CU2altadeusuario}
		\UCpaso[\UCactor] Termina de llenar los datos del formulario y presiona el boton de aceptar. 
		\UCpaso Muestra el {\bf MSG1-} ''Nuevo componente agregado al perfil''%% \BRref{BR130}{Determinar si un Estudiante puede inscribirse en un Seminario} \Trayref{C}.
		\UCpaso Muestra al usuario una ventana para preguntar si desea agregar otro componente o regresar al pagina principal \Trayref{B} \Trayref{C}
		
%%		\UCpaso Direcciona al usuario a la pantalla principal 
			
	\end{UCtrayectoria}
		
		\begin{UCtrayectoriaA}{A}{El usuario no escribe en todos los campos}
			\UCpaso Muestra el Mensaje {\bf MSG2-}``El campo [{\em nombre del campo a llenar}] no puede quedar vacio.''.
			\UCpaso[\UCactor] Oprime el botón \IUbutton{Aceptar}.
			\UCpaso Solicita llenar el campo requerido. 
			\UCpaso Continua en el paso \ref{CU2altadeusuario} del \UCref{CU3}.
			\UCpaso[] Termina el caso de uso.
		\end{UCtrayectoriaA}
		
		\begin{UCtrayectoriaA}{B}{El usuario desea agregar otro componente}
			\UCpaso[\UCactor] Presiona el boton de agregar otro componente \IUbutton{Salir}.
			\UCpaso Se regresa al formulario de agregar componente.\ref{CU2alta} del \UCref{CU3}.
			%\UCpaso  Muestra el Mensaje {\bf MSG4-}''La contrase\~na insertada es invalida.''.
			%\UCpaso[\UCactor] Oprime el botón \IUbutton{Salir}.
			%\UCpaso Solicita una nueva contrase\~na. 
			%\UCpaso Continua en el paso \ref{CU2altadeusuario} del \UCref{CU2}.
			\UCpaso[] Termina el caso de uso.
		\end{UCtrayectoriaA}

		\begin{UCtrayectoriaA}{B}{El usuario no desea agregar otro componente}
			\UCpaso[\UCactor] Presiona el boton de regresar a la pagina principal \IUbutton{Salir}.
			\UCpaso Regresa a la pagina principal.			
%			\UCpaso  Muestra el Mensaje {\bf MSG4-}''La contrase\~na insertada es invalida.''.
%			\UCpaso[\UCactor] Oprime el botón \IUbutton{Salir}.
%			\UCpaso Solicita una nueva contrase\~na. 
			\UCpaso Continua en el paso \ref{CU2altadeusuario} del \UCref{CU2}.
			\UCpaso[] Termina el caso de uso.
		\end{UCtrayectoriaA}
		
%-------------------------------------- TERMINA descripción del caso de uso.