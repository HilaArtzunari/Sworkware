% \IUref{IUAdmPS}{Administrar Planta de Selección}
% \IUref{IUModPS}{Modificar Planta de Selección}
% \IUref{IUEliPS}{Eliminar Planta de Selección}

% 


% Copie este bloque por cada caso de uso:
%-------------------------------------- COMIENZA descripción del caso de uso.

%\begin{UseCase}[archivo de imágen]{UCX}{Nombre del Caso de uso}{
	\begin{UseCase}{CU5}{Editar componente}{
		El estudiante podrá editar los componentes electrónicos que tenga dados de alta en su cuenta de usuario para actualizar su información.
	}
		\UCitem{Versión}{0.1}
		\UCitem{Actor}{Alumno}
		\UCitem{Propósito}{Actualizar la información del componente electrónico cuando haya cambiado}
		\UCitem{Resumen}{
		El sistema muestra la información actual del componente electrónico para que el estudiante pueda elegir editarla y así actualizar su contenido}
		\UCitem{Entradas}{
		- Foto del componente electrónico: Imagen con formato .jpg
		
		- Nombre: Cadena de caracteres
		
		- Cantidad: número entero positivo
		
		- Precio: número de coma flotante positivo.
		
		- Descripción: Cadena de caracteres acerca del componente.
		
		- Link a la datasheet (opcional): URL con el formato http://www.example.com
		}
		\UCitem{Salidas}{Mensaje {\bf MSG-19}``Se ha actualizado la información del componente correctamente''}
		\UCitem{Precondiciones}{
		- El componente debe estar registrado.
		
		- El usuario requiere tener una cuenta.
		
		- La cuenta de usuario no puede estar bloqueada.
		
		- Conexión a internet.
		}
		\UCitem{Postcondiciones}{El componente se actualizará.}
		\UCitem{Autor}{Gabriela Moreno González}
		\UCitem{Revisor}{Angel Ferreira Osorno}
	\end{UseCase}

	\begin{UCtrayectoria}{Principal}
		\UCpaso[\UCactor] Da clic en el botón \IUbutton{Editar componente} en la \IUref{UI19}{Ver componentes disponibles}
		\UCpaso[\UCactor] Introduce los datos del componente electrónico en la \IUref{UI20}{Editar información del componente}
		\UCpaso[\UCactor] Da clic en el botón \IUbutton{Guardar información} \Trayref{A} \Trayref{B}
		\UCpaso Actualiza la información del componente electrónico.
		\UCpaso Muestra el mensaje 'Se ha actualizado la información del componente correctamente' 
	\end{UCtrayectoria}
		
		\begin{UCtrayectoriaA}{A}{Uno o más datos no fueron ingresados}
			\UCpaso Muestra el Mensaje {\bf MSG-20}``Por favor ingresa todos los campos''
			\UCpaso[\UCactor] Oprime el botón \IUbutton{Aceptar}.
			\UCpaso[\UCactor] Continúa en el paso 2 del \UCref{CU5}.
		\end{UCtrayectoriaA}
		
		\begin{UCtrayectoriaA}{B}{Los formatos de los datos son incorrectos}
			\UCpaso Muestra el mensaje {\bf MSG-21}``Por favor ingresa los campos correctamente''
			\UCpaso[\UCactor] Oprime el botón \IUbutton{Aceptar}
			\UCpaso Continua en el paso 2 del \UCref{CU5}.
		\end{UCtrayectoriaA}

%-------------------------------------- TERMINA descripción del caso de uso.