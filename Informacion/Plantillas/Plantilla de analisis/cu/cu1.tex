% \IUref{IUAdmPS}{Administrar Planta de Selección}
% \IUref{IUModPS}{Modificar Planta de Selección}
% \IUref{IUEliPS}{Eliminar Planta de Selección}

% 


% Copie este bloque por cada caso de uso:
%-------------------------------------- COMIENZA descripción del caso de uso.

%\begin{UseCase}[archivo de imágen]{UCX}{Nombre del Caso de uso}{
	\begin{UseCase}{CU1}{Login}{
		Permite al usuario identificarse dentro del sistema.
	}
		\UCitem{Versión}{0.1}
		\UCitem{Actor}{Administrado, usuario}
		\UCitem{Propósito}{Identificación en el sistema.}
		\UCitem{Resumen}{El sistema solicitara el usuario realizar una identificación para poder otorgarle las funciones necesarias.}

		\UCitem{Entradas}{E-mail: Cadena de caracteres con el siquiente formato xxxx@dominio.com
		Donde xxxx equivale al nombre de usuario, @dominio.com equivale al proveedor de la cuenta.
		password: Cadena de caracteres con un minimo de 8 letras, como m\'aximo 15 letras, entre ellas debe haber al menos un n\'umero. }
		\UCitem{Origen}{Teclado.}		
		
		\UCitem{Salidas}{MSG1 Bienvenido nombre usuario donde nombre usuario sera el nombre del usuario correspondiente.}
		
		\UCitem{Destino}{Pantalla del usuario.}

		\UCitem{Precondiciones}{Debe existir una cuenta relacionada con el user y el password para poder realizar la identificación. También deben ser escritas correctamente para poder tener acceso al sistema.}

		\UCitem{Postcondiciones}{Se realizar\'a una comprobaci\'on para dar acceso al catalogo.}
		
		\UCitem{Tipo}{Primario.}

		\UCitem{Observaciones}{:).}
						
				
		\UCitem{Autor}{Luis \'Angel Corona Elizalde.}

		\UCitem{Revisor}{Gaby.}
		
	\end{UseCase}		

	\begin{UCtrayectoria}{Principal}
		\UCpaso[\UCactor] El usuario ingresa al sistema.  \IUref{UI23}{Pantalla principal del sistema}\label{CU17Login}.
		\UCpaso[\UCactor] El usuario ingresa los datos necesarios para tener acceso al sistema, en este caso debe de ingresar e-mail y password.\Trayref{B}\label{CU1login} %\BRref{BR129}{Determinar si un Estudiante puede inscribir Seminario.} \Trayref{A}.
		\UCpaso[\UCactor] Ya escritos los datos necesarios y correctos presiona el boton de ingresar al sistema.% Despliega la \IUref{UI32}{Pantalla de Selección de Seminario} con la lista de Seminarios Disponibles.
		\UCpaso El sistema realiza la validación del e-mail y del password insertados.\Trayref{A}\Trayref{B}%%Selecciona el Seminario en el que desea inscribirse \Trayref{B}\label{CU17SeleccionarSeminario}.
		\UCpaso Muestra el {\bf MSG1-} "Bienvenido"%% \BRref{BR130}{Determinar si un Estudiante puede inscribirse en un Seminario} \Trayref{C}.
		\UCpaso Direcciona al usuario a la pantalla principal %% \BRref{BR143}{Validar el horario del estudiante} \Trayref{D}.
%		\UCpaso Calcula el costo del Seminario basado en el costo publicado en el catálogo de cursos, los costos aplicables al alumno y los impuestos aplicables, con base en las reglas \BRref{BR180}{Calcular costos del Estudiante} y \BRref{BR45}{Calcular impuestos por seminario}.
		
%		\UCpaso Despliega el desglose de costos en la \IUref{UI33}{Pantalla Mostrar costos por seminario}.
%		\UCpaso Pide al Estudiante que confirme la inscripción alSeminario.
%		\UCpaso[\UCactor] Confirma la inscripción al Seminario.
%		\UCpaso Inscribe al Estudiante en el Seminario seleccionado.
%		\UCpaso Informa que la inscripción se realizó exitosamente vía la \IUref{UI88}{Pantalla de resumen de inscripción al Seminario}. 
%		\UCpaso Imprime el recibo de pago con base en la regla \BRref{BR100}{Recibo del Estudiante por inscripción a Seminario.}.
%		\UCpaso Pregunta al estudiante si desea imprimir un comprobante de la inscripción.
%		\UCpaso[\UCactor] Indica que desea imprimir el comprobante de la inscripción.
%		\UCpaso Imprime el comprobante de la inscripción \IUref{UI189}{Reporte de inscripción a Seminario}.		
	\end{UCtrayectoria}
		
		\begin{UCtrayectoriaA}{A}{El usuario no puede ingresar al sistema ''contrase\~na incorrecta''}
			\UCpaso Muestra el Mensaje {\bf MSG2-}``El password del usuario [{\em nombre de usuario}] es incorrecta.''.
			\UCpaso[\UCactor] Oprime el botón \IUbutton{Aceptar}.
			\UCpaso Regresa a la pantalla de inicio. 
			\UCpaso Continua en el paso \ref{CU1login} del \UCref{CU1}.
			\UCpaso[] Termina el caso de uso.
		\end{UCtrayectoriaA}
		
		\begin{UCtrayectoriaA}{B}{El usuario no existe}
			\UCpaso  Muestra el Mensaje {\bf MSG3-}``El usuario [{\em nombre de usuario}] es invalido.''.
			\UCpaso[\UCactor] Oprime el botón \IUbutton{Salir}.
			\UCpaso Regresa a la pantalla de inicio. 
			\UCpaso Continua en el paso \ref{CU1login} del \UCref{CU1}.
		\end{UCtrayectoriaA}

%		\begin{UCtrayectoriaA}{C}{El estudiante no cumple con los prerrequicitos}
%			\UCpaso Muestra el Mensaje {\bf MSG2-}``El Estudiante [{\em Número de Boleta}] no cumple con los requisitos para inscribirse al Seminario [{\em Nombre del Seminario seleccionado}].''.
%			\UCpaso Muestra los requisitos que el Seminario seleccionado solicita.
%			\UCpaso Continúa en el paso \ref{CU17SeleccionarSeminario} del \UCref{CU17}.
%		\end{UCtrayectoriaA}

%		\begin{UCtrayectoriaA}{D}{El horario es incompatible.}
%			\UCpaso Muestra el Mensaje {\bf MSG3-}``El horario del [{\em Nombre del Seminario seleccionado}] no es compatible con el horario del curso [{\em Nombre de la materia y grupo del curso con el que choca el horario}].''.
%			\UCpaso Continúa en el paso \ref{CU17SeleccionarSeminario} del \UCref{CU17}.
%		\end{UCtrayectoriaA}
		
%-------------------------------------- TERMINA descripción del caso de uso.
