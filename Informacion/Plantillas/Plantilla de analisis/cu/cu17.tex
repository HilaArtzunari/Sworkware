\begin{UseCase}{CU17}{Cambiar contraseña}{
	Permite al actor cambiar su contraseña actual por una nueva, ya que tuvo algún problema de seguridad o simplemente quiere modificarla, para realizar el cambio el sistema requiere que se ingrese su contraseña actual ya que es por motivos de seguridad y saber que si es el usuario quien lo solicite.
}

	\UCccitem{Versión}{0.1}	
		\UCitem{Actor}{Usuario, Administrador}
		\UCitem{Propósito}{Cambiar la contraseña por motivos de seguridad del actor.
		}
		\UCitem{Entradas}{ 
			\begin{Titemize}
				\Titem Contraseña actual.	
				\Titem Contraseña nueva.
				\Titem Repetir contraseña nueva.
			\end{Titemize}
			 }
		\UCitem{Origen}{Mouse}
		\UCitem{Salidas}{
			\begin{Titemize}
			\Titem \refIdElem{MSGX}
			\Titem \refIdElem{MSGX}
			\Titem \refIdElem{MSGX}
			\end{Titemize}

		}
		\UCitem{Destino}{Pantalla}
		\UCitem{Precondiciones}{
			Estar registrado en el sistema.
	
		}
		\UCitem{Postcondiciones}{
		Se guardará en el sistema su nueva contraseña.
	}
		\UCitem{Reglas de Negocio}{
		\begin{Titemize}
			\Titem \refIdElem{BR-N00X}
			\Titem \refIdElem{BR-N00X}
			\Titem \refIdElem{BR-N00X}
		\end{Titemize}	
	}
		\UCitem{Errores}{\begin{Titemize}
			\Titem \UCerr{Uno}{{Cuando las contraseñas \textbf{Nueva} y \textbf{Repetir contraseña nueva} no coinciden,}{ muestra el mensaje \refIdElem{MSGX}, indicando que la contraseñas no coinciden.}
			\Titem \UCerr{Dos}{Cuando el actor no ingresó su contraseña \textbf{Actual}, }{muestra la el mensaje \refIdElem{MSGX}, indicando que la contraseña no se ingresó.}
		\end{Titemize}
		}}
		\UCitem{Viene de}{\refIdElem{CU1}}
		\UCitem{Disparadores}{
			\begin{Titemize}
			\Titem Se requiere cambiar la contraseña.
			\Titem Olvido su contraseña.
			\end{Titemize}
		}
		\UCitem{Condiciones de Término}{
			Se realizó el cambio de contraseña.
		}
		\UCitem{Efectos Colaterales}{Ninguno}
		\UCitem{Referencia Documental}{C1-PF Proceso Fortalecido}
	
\end{UseCase}

\begin{UCtrayectoria}

	\UCpaso [\UCactor]Solicita cambiar su contraseña presionando el botón \IUbutton{Editar} de Cambiar Contaseña, de la pantalla \refIdElem{IU17a}.
	
	\UCpaso Muestra la pantalla \refIdElem{IU17b}.
	
	\UCpaso [\UCactor] Ingresa en los campos la información solicitada.  \refTray{A} \refTray{B}
	
	\UCpaso Habilita el botón \IUbutton{Guardar cambios}. 
	
	\UCpaso [\UCactor] Presiona el botón \IUbutton{Guardar cambios}. \refTray{C}
	
	\UCpaso Verifica que en el campo marcado como \textbf{Actual} haya ingresado la contraseña actual  con base en la regla de negocio \refIdElem{BR-S00X}.  \refTray{D}
	
	\UCpaso Verifica que los datos ingresados en los campos de \textbf{Nueva} y \textbf{Repetir contraseña} nueva sean los mismos con base en la regla de negocio \refIdElem{BS-N00X}. \refTray{D}

	\UCpaso  Muestra el mensaje \refIdElem{MSGX}, indicando que la operación ha finalizado con con exito.
	
\end{UCtrayectoria} 

\begin{UCtrayectoriaA}[Fin de la trayectoria]{A}{Cuando el actor ha olvidado su contraseña.}
	
	\UCpaso Muestra la el mensaje \refIdElem{MSGX}, indicando que si desea continuar mandará un correo con su contraseña actual. 
	
	\UCpaso [\UCpaso] Presiona el botón \IUbutton{Continuar}. \refTray{C}
	
	\UCpaso Envía un correo electrónico a la cuenta del actor que solicita el cambio de contraseña.	
	
	\UCpaso [\UCpaso] Ingresa en el campo la información que recibió en su correo electrónico.
	
	\UCpaso Verifica que la información ingresada en el campo coincida con la que se mando en el correo electrónico. \refTray{D}

	\UCpaso Presiona el botón \IUbutton{Continuar}. \refTray{C}

	\UCpaso Regresa a la pantalla \refIdElem{IU17b}.
	
\end{UCtrayectoriaA}

\end{UCtrayectoriaA}

\begin{UCtrayectoriaA}[Fin de la trayectoria]{B}{Cuando el actor no ingresa información en los campos.}
	
	\UCpaso No habilita el botón \IUbutton{Guardar cambios}.
	
	\UCpaso Regresa al paso 3 de la trayectoria principal.
	
\end{UCtrayectoriaA}

\begin{UCtrayectoriaA}[Fin de la trayectoria]{C}{Cuando el actor desea cancelar la operación.}
	
	\UCpaso [\UCpaso] Solicita cancelar la operación presionando el botón \IUbutton{Cancelar}.
	
	\UCpaso Muestra la pantalla \refElem{IU17a}.
	
\end{UCtrayectoriaA}

\begin{UCtrayectoriaA}[Fin de la trayectoria]{D}{Cuando la información en un campo no coincide con lo que estaba establecido.}
	
	\UCpaso Muestra el mensaje \refIdElem{MSGX}.	
	
	\UCpaso Regresa al paso 3 de la pantalla principal.
	
\end{UCtrayectoriaA}