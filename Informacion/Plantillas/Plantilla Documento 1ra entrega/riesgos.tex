\chapter{Planificación de la gestión de riesgos}	
\label{cap:alcance}
 
Se definirá como realizar las actividades para la gestión de riegos del proyecto, es importante para proporcionar los recursos y el tiempo suficiente para dichas actividades.

\section{Gestión de riesgos}
Es necesario el planteamiento y definición del equipo encargado de la gestión de riesgos, el cual contará con la designación de tareas y actividades puntuales que estén enfocadas al cumplimiento y progreso de la estructura.

\begin{table}[hbtp!]
	\noindent\begin{tabular}{|p{.4\textwidth}|p{.6\textwidth}|}
		\hline
		{\bf Responsable} & {\bf Funciones}\\
		\hline
		Jefe del proyecto (Moreno González Gabriela) & Tiene como función el autorizar recursos para la mitigación, integrar información de todos los responsables de área sobre los riesgos,
		revisar la prioridad para determinar cuáles son los riesgos importantes, tomar decisiones de control, asignar y cambiar 	responsabilidades dentro del proyecto. \\
		\hline	 
		Equipo Interno de Gestión del riesgo (Mejía Mendoza Diana Laura, Mendoza Saavedra Roberto) & Debe coordinar actividades para identificar y analizar riesgos, mantener la lista de los riesgos del proyecto, notificar nuevos riesgos e informar periódicamente sobre el estado de los riesgos al jefe del proyecto. \\
		\hline
		Equipo de soporte a la Gestión del riesgo(Ferreira Osorno Ángel, Corona Luis Ángel) & Deben detectar elementos de riesgo y estimar su impacto potencial negativo. \\
		\hline
	\end{tabular}
	\caption{Definición del equipo de Gestión de Riesgos.}
	\label{tbl:DefinicionEquipoGestionRiegos}
\end{table}



\section{Análisis de riesgos}

Se realiza una reunión para la identificación de los riesgos que se pudieran presentar antes de la ejecución, tomando en cuenta la opinión del personal, un análisis técnico de las actividades.
Una vez identificados los principales riesgos que pueden tener impactos significativos, y haber analizado sus posibles consecuencias y causas, se procede a determinar la probabilidad
de ocurrencia, ya sea de amenaza o de oportunidad en el desarrollo de la implementación del software.
\begin{table}[hbtp!]
	\noindent\begin{tabular}{|p{.3\textwidth}|p{.15\textwidth}|}
		\hline
		{\bf Probabilidad} & {\bf Escala relativa}\\
		\hline
		Muy alta & 0.9 \\
		\hline	 
		Alta & 0.7 \\
		\hline
		Moderada & 0.5 \\
		\hline
		Baja & 0.3 \\
		\hline
		Muy baja & 0.1 \\
		\hline
	\end{tabular}
\begin{flushleft}
		\caption{Definición de probabilidad.}
\end{flushleft}
	\label{tbl:DefinicionProbabilidad}
\end{table}

\begin{table}[hbtp!]
	\noindent\begin{tabular}{|p{.3\textwidth}|p{.15\textwidth}|}
		\hline
		{\bf Impacto} & {\bf Escala relativa}\\
		\hline
		Catastrófico & 0.9 \\
		\hline	 
		Mayor & 0.7 \\
		\hline
		Moderado & 0.5 \\
		\hline
		Menor & 0.3 \\
		\hline
		Insignificativo & 0.1 \\
		\hline
	\end{tabular}
\begin{flushleft}
		\caption{Definición de impacto.}
\end{flushleft}
	\label{tbl:DefinicionImpacto}
\end{table}

\begin{table}[hbtp!]
	\noindent\begin{tabular}{|p{.2\textwidth}|p{.4\textwidth}|p{.2\textwidth}|p{.2\textwidth}|}
		\hline
		{\bf Riesgos} & {\bf Descripción} & {\bf Impacto(I)} & {\bf Probabilidad(P)} \\
		\hline
		Falta de comunicación entre miembros del staff & En caso que la comunicación entre los miembros del staff no sea adecuada o no se produzca a tiempo y esto
		tenga consecuencias en demoras en actividades o falta de información. & Menor & Baja  \\
		\hline	 
		No contemplar todos los requerimientos de usuario y/o sistema & Si no se contemplan todos los requerimientos del usuario, no se tendrían en cuenta los requerimientos del sistema y por lo tanto no se localizarían reglas de negocio involucradas y sus solicitudes. & Moderado & Moderada \\
		\hline
		Desacuerdos entre involucrados & Puede darse el caso que los involucrados no se lleguen a poner de acuerdo al momento de tomar decisiones importantes sobre el análisis, desarrollo o implementación del proyecto. & Insignificativo  & Muy baja \\
		\hline
		Problemas técnicos con los dispositivos con que se trabaja & Si en las actividades técnicas programadas en la estructura del trabajo a realizarse, el tiempo de ejecución de las mismas excede de manera significativa el estipulado debido a cualquier complicación de índole técnica que pueda surgir. & Menor & Baja \\
		\hline
		Integrantes del staff abandonan el proyecto antes que finalice & Si por algún motivo un integrante del staff abandona el proyecto antes que finalice, el tiempo de cada tarea para llevar a cabo el proyecto seria mayor provocando complicaciones en el proyecto & Mayor & Baja  \\
		\hline
		El personal no cuenta con los conocimientos requeridos para enfrentar la complejidad del requisito & Cuando un miembro del equipo no cuenta con los conocimientos necesarios, las tareas del proyecto pueden demorar mas de lo provisto. & Moderado & Baja \\
		\hline
		Miembros del equipo no disponibles en momentos críticos & Los miembros de staff pueden tener problemas familiares, enfermedades o algún problema que pueda afectar el avance del proyecto en algún momento critico. & Mayor & Moderada  \\
		\hline
		El tiempo requerido para desarrollar el proceso de requisitos está subestimado & Todas las peticiones o requerimientos produce un riesgo de demorar más de lo previsto inicialmente y tener repercusiones en el
		desarrollo del proyecto. & Catastrófico & Alta\\
		\hline
	\end{tabular}
	\caption{Análisis de Riesgos.}
	\label{tbl:AnalisisRiegos}
\end{table}
