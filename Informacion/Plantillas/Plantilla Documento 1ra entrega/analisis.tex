%=========================================================
\chapter{Análisis del problema}
\label{cap:analisis}

Este capítulo contiene la justificación del proyecto. Presenta una breve descripción de las materias del Politécnico, un análisis del problema identificando sus principales causas y termina con la estimación de las consecuencias más importantes a mediano y largo plazo.

%----------------------------------------------------------
\section{Términos del negocio}
\label{sec:terminosDeNegocio}
% Status:
% \cdtStRevision cuando el autor ya terminó de trabajar
% \cdtStRevised  cuando el revisor acepta la redacción
% \cdtStEnded    cuando la revisión fue aprobada por el líder de proyecto
% \cdtStAccepted cuando el Cliente aprobó la redacción
\begin{brGlosario}[
	version=1.0, 
	author=Gabriela Moreno, 
	revisor=Ulises Pérez, 
	status =\cdtStEdition
]
	\brTermType{tCoordinador}{Coordinador:} ...
	\brTermType{tIntegrante}{Integrante:} ...
	\brTermType{tParticipante}{Participante:} ...
	\brTermType{tActividad}{Actividad:} ...

%	\brTermLiteral{tAutomovil}{Automóvil:}{tVehiculo}{Vehículo} De cuatro ruedas con capacidad de 5 a 9 personas. 
%	\brTermType{tCliente}{Cliente:} Se refiere a todas las personas físicas y morales que \cdtRef{tRenta}{rentan} o han rentado un \cdtRef{tVehiculo}{vehículo}.
%	
%	\brTermLiteral{tDirector}{Director:}{tEmpleado}{Empleado} Es el empleado que tiene mayor rango de todos y no tiene superior, a diferencia de los demás.
%	
%	\brTermType{tEmpleado}{Empleado:} Se refiere a cualquier persona que labore en la empresa.
%	
%	\brTermClock{tChecador}{Checador:}{Hora de entrada y salida de un \cdtRef{tEmpleado}{empleado}.}{Una vez al día para la entrada y otra para la salida durante los días laborales.}
%	
%	\brTermLiteral{tMotocicleta}{Motocicleta:}{tVehiculo}{Vehículo} De dos ruedas con capacidad para una personas. 
%
%	\brTermType{tRenta}{Renta:} Se refiere al servicio que ofrece la empresa para prestar \cdtRef{tVehiculo}{vehículos} a los \cdtRef{tCliente}{clientes} por un tiempo definido.
%	
%	\brTermType{tVehiculo}{Vehiculo:} Se refiere a los automóviles y motocicletas que la empresa usa para dar el servicio de renta a los \cdtRef{tCliente}{clientes}.
%	
%	\brTermSensor{tVelocimetro}{Velocímetro:}{Velocidad de un Vehículo.}{Kilometros/hora.}{Constantemente siempre que el \cdtRef{tVehiculo}{vehículo} esté encendido.}
\end{brGlosario}



%---------------------------------------------------------
\section{Descripción del contexto}

En la ESCOM - IPN se imparten Unidades de Aprendizaje de Electrónica como 'Análisis fundamental de circuitos', 'Instrumentación', entre otras, en las cuales se le solicita al alumno adquirir una serie de materiales para poder realizar circuitos que le permitan comprender el funcionamiento de algunas leyes físicas y de dichos componentes, sin embargo algunos de éstos no son fáciles de encontrar y una vez que acreditan la U.A pueden no volver a requerir usar el material, por ello algunos publican en redes sociales que están vendiendo dichos materiales y otros publican que buscan a alguien que los vendan, mezclando este interés de adquirirlos con otras publicaciones, haciendo difícil que los interesados se comuniquen de forma efectiva.
	
%---------------------------------------------------------
\section{Descomposición del problema}

	Actualmente las escuelas superiores del IPN no cuentan con una manera eficiente de adquirir su material electrónico. Los problemas más comunes al ir a adquirir un material electrónico son los siguientes:
	
\begin{itemize}
	\item Ninguna tienda tiene el componente
	\item El componente ha dejado de ser fabricado
	\item El componente es muy caro
	\item El componente no tiene especificaciones
	\item El componente viene dañado y no posee garantía
\end{itemize}

%---------------------------------------------------------
\section{Análisis del problema}
\subsection{Problemática}

Los estudiantes de escuelas superiores del IPN en Zacatenco requieren vender y comprar material electrónico para sus Unidades de Aprendizaje, sin embargo pueden no encontrar el material que requieren o que éste sea muy caro.

\subsection{Descomposición}

No existe una plataforma que permita a los estudiantes la búsqueda eficaz de su material, así como el hecho de que algunos materiales solicitados están descontinuados en el mercado y su adquisición es casi imposible. 

Los componentes comprados hace años por otros estudiantes o jamás se ponen en venta o simplemente los publican en sitios poco convenientes como Facebook y la publicación jamás es vista por el público que está interesado. 

La única forma en la que algunos estudiantes regalan sus componentes es cuando conocen a la persona y le solicita que se los preste, a lo que el estudiante prefiere regalarlos porque no los va a volver a usar. 

Muchas veces los estudiantes realizan la inversión para comprar sus materiales y nunca los vuelven a usar ni en el resto de su carrera ni cuando comienzan a laborar.

\subsection{Identificación de causas.}

Algunas de las causas del problema son:

\begin{enumerate}
	\item Nadie ha propuesto un sistema para comprar y venta de material electrónico.
	\item Algunos estudiantes cuentan con el nivel económico para pagar materiales caros.
	\item En Zacatenco no se cuenta con una área exclusiva para guardar material electrónico para su préstamo y devolución.
	\item Las propuestas hechas en algunas escuelas no han podido desarrollarse debido a la falta de apoyo por parte de autoridades del IPN.
\end{enumerate}

\subsection{Estimación de consecuencias}
	
Las consecuencias ya observables son:

\begin{enumerate}
	\item Los estudiantes se ven afectados económicamente al no poder revender su material.
	\item Los estudiantes se ven afectados de manera escolar cuando no consiguen el material que requieren para realizar sus prácticas.
	\item Los profesores se ven afectados en la elaboración de las prácticas si el material que van a solicitar es caro o de difícil adquisición.
	\item Los estudiantes se ven afectados en su manera de aprender cuando no pueden experimentar con diversos componentes dentro de la carrera.
	\item Los estudiantes corren el riesgo de ser timados al comprar un material por desconocer su precio real.
	\item Los estudiantes pueden ser afectados en sus gastos cuando requieren invertir mucho dinero solo porque el material es nuevo.		
\end{enumerate}

%---------------------------------------------------------
\section{Síntesis y propuesta de solución}

\subsection{Soluciones existentes}

Las soluciones parciales que se han hecho son las siguientes:

\begin{enumerate}
	\item Publicar el material en venta en redes sociales.
	\item Prestarse el material con conocidos.
	\item Comprar el material en el centro o en SISCOM.
\end{enumerate}

% - - - - - - - - - - - - - - - - - - - - - - - - - - - - - 
\subsection{Propuesta de solución}

Realizar un sistema que permita la administración de ventas y donaciones de componentes electrónicos entre los estudiantes de escuelas superiores del IPN en Zacatenco.

Las características del sistema son:

\begin{itemize}
	\item  Poseerá un repositorio de material electrónico.
	\item Los estudiantes podrán crear un perfil en el sistema.
	\item  Se tendrá el link al perfil de Facebook de los estudiantes con un perfil en el sistema
	\item Los estudiantes podrán donar, vender o comprar componentes
	\item  Los componentes tendrán la siguiente información: nombre, cantidad, link a la datasheet (opcional) y si se va a donar o a vender
	\item  Se podrán calificar los perfiles de los usuarios
	\item  El sistema será un sitio web.
	\item Se empleará BootStrap para su compatibilidad con dispositivos móviles
\end{itemize}
